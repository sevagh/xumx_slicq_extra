\documentclass[letter,12pt]{article}
\usepackage[left=2cm, right=2cm, top=2cm, bottom=2cm]{geometry}
\usepackage[shortlabels]{enumitem}
\usepackage{graphicx}
\usepackage{todonotes}
\usepackage{amsmath}
\usepackage[titletoc,title]{appendix}
\usepackage{amssymb}
\usepackage{makecell}
\usepackage{wrapfig}
\usepackage{verbatim}
\usepackage{listings}
\usepackage{minted}
\usepackage{subfig}
\usepackage{titling}
\usepackage[compatibility=false]{caption}
\usepackage[parfill]{parskip}
\setlength{\droptitle}{1cm}
\usepackage{hyperref}
\hypersetup{
    colorlinks,
    citecolor=black,
    filecolor=black,
    linkcolor=black,
    urlcolor=black
}
\usepackage{setspace}
\renewcommand{\topfraction}{0.85}
\renewcommand{\textfraction}{0.1}
\renewcommand{\floatpagefraction}{0.75}
\usepackage[
    %backend=biber, 
    natbib=true,
    style=numeric,
    sorting=none,
]{biblatex}
\addbibresource{citations.bib}
\newcommand\ThesisTitle{Music Source Separation with the sliCQ Transform}


\setcounter{biburllcpenalty}{7000}
\setcounter{biburlucpenalty}{8000}

\newenvironment{tight_enumerate}{
\begin{enumerate}
  \setlength{\itemsep}{0pt}
  \setlength{\parskip}{0pt}
}{\end{enumerate}}

\newenvironment{tight_itemize}{
\begin{itemize}
  \setlength{\itemsep}{0pt}
  \setlength{\parskip}{0pt}
}{\end{itemize}}

\newlength{\mintednumbersep}
\AtBeginDocument{%
  \sbox0{\tiny00}%
  \setlength\mintednumbersep{8pt}%
  \addtolength\mintednumbersep{-\wd0}%
}

\title{\ThesisTitle}

\author{\vspace{2em}\\Sevag Hanssian \\
  McGill University \\
 \texttt{sevag.hanssian@mail.mcgill.ca} \\
 \texttt{sevagh@protonmail.com} \\\ \\\ \\
 Thesis for Master of Arts in Music Technology}

\date{}

\begin{document}

\maketitle

\vfill
\clearpage %force a page break

\tableofcontents

\vfill
\clearpage %force a page break

\listoffigures

\listoflistings

\vfill
\clearpage %force a page break

\begin{abstract}
	The discrete Fourier transform (DFT), along with the Fast Fourier Transform (FFT) \cite{cooleytukey} algorithm for its efficient computation, rank among the most important algorithms in applied engineering and computer science \cite{dftimportant}. The DFT is used in digital signal processing applications to decompose a discrete-time acoustic signal into a sum of its frequency components, generating a spectrum to perform what is known as spectral or frequency analysis. This Master's thesis first proposes to describe and implement the sparse, pruned, nonuniform, warped, and fractional variants of the DFT or FFT. Next, the use of the standard DFT and FFT in real-world music systems will be revisited to explore whether using the variants can be beneficial. The main result achieved is that a combination of several of the variants showed a reduction in computational footprint and improvement in accuracy in a recent state-of-the-art music system\todo{REAL-WORLD SOTA}.
\end{abstract}

\section{Introduction}
\label{sec:intro}

   Despite its importance, the DFT has some deficiencies in real-world signals. First, the human auditory system has a nonlinear frequency resolution, with finer frequency discrimation at low frequencies \todo{cite this}, but the DFT outputs linearly-spaced frequencies. Second, real-world signals are typically sparse, with most of their energy concentrated in few frequency components \cite{sparseintro}, yet the DFT outputs values for many frequencies, most of which are discarded.

\subsection{Motivation}

Music and signal processing literature often show promising results by using variants of the Fourier transform\todo{cite proposal shit here}.

However, implementations of these variants are not readily found or available in common DFT/FFT software libraries. At the time of writing this thesis, fftw,\footnote{\href{http://www.fftw.org/}{http://www.fftw.org/}} ffts,\footnote{\href{https://github.com/anthonix/ffts/}{https://github.com/anthonix/ffts/}} numpy,\footnote{\href{https://numpy.org/doc/stable/reference/routines.fft.html}{https://numpy.org/doc/stable/reference/routines.fft.html}} scipy,\footnote{\href{https://docs.scipy.org/doc/scipy/reference/tutorial/fft.html}{https://docs.scipy.org/doc/scipy/reference/tutorial/fft.html}} and hardware-specific libraries such as Intel Performance Primitives (IPP),\footnote{\href{https://software.intel.com/content/www/us/en/develop/documentation/ipp-dev-reference/top/volume-1-signal-and-data-processing/transform-functions/fourier-transform-functions/fast-fourier-transform-functions.html}{https://software.intel.com/content/www/us/en/develop/documentation/ipp-dev-reference/top/volume-1-signal-and-data-processing/transform-functions/fourier-transform-functions/fast-fourier-transform-functions.html}} cuFFT (CUDA FFT for NVIDIA GPUs),\footnote{\href{https://docs.nvidia.com/cuda/cufft/index.html}{https://docs.nvidia.com/cuda/cufft/index.html}} and Ne10 for ARM processors\footnote{\href{https://projectne10.github.io/Ne10/doc/group__groupDSPs.html}{https://projectne10.github.io/Ne10/doc/group\_\_groupDSPs.html}} only contain the standard FFT algorithms. New FFT libraries are written regularly, citing even better performance from taking advantage of new processor instructions,\footnote{\href{https://users.rust-lang.org/t/rustfft-5-0-0-experimental-1-now-faster-than-fftw/53049}{https://users.rust-lang.org/t/rustfft-5-0-0-experimental-1-now-faster-than-fftw/53049}} but few of them implement more than the standard FFT.

\subsection{Thesis objective}

The primary objective of this thesis is to collect implementations of different variants of the discrete Fourier transform (DFT) and fast Fourier transform (FFT) in a single library with a well-defined interface that resembles the traditional DFT or FFT as closely as possible. The intention is to be able to be substitute any of the variants in existing codebases with minor (or ideally, no) code changes, to reduce the friction of adoption. For maximum compatibility, the core library is written in C, with bindings created for Python. The majority of higher-level tools and languages used for MIR and DSP applications have capabilities for using C libraries, so that the fftn would only need to have additional wrappers or bindings created to support arbitrary languages. \todo{need C universal abi citation}

The secondary objective of this thesis is to substitute the implemented variants in a state-of-the-art music system and determine whether 

\subsection{Related work}

foo

\subsection{Contribution and results}

My intended and favorite contribution of this thesis is the creation and open-source publication of a new Fourier transform library,\footnote{\href{https://github.com/sevagh/fftn}{https://github.com/sevagh/fftn}} named ``fftn,'' which stands for a \textit{Fancy Fourier Transforms in the North}. The name is chosen in the same spirit as fftw\cite{fftw} (Fastest Fourier Transform in the West), created at the Massachusetts Institute of Technology (MIT) in the United States and named after the ``Fastest Gun in the West'' trope from Spaghetti Western movies,\footnote{\href{http://www.fftw.org/faq/section1.html\#west}{http://www.fftw.org/faq/section1.html\#west}} and ffts\cite{ffts} (Fastest Fourier Transform in the South), created at The University of Waikato in New Zealand which is located in the southern hemisphere of the world. The goal of my library is not speed through further optimizing the standard FFT, but to gather many variants of the FFT (which may inherently include a speedup) in a single place, intended to be explored and experimented with as drop-in replacements of the traditional FFT in music systems. Hence, the word \textit{Fancy} denotes the implementation of interesting and uncommon variants, and north is because I created it in my city of Montr{\'e}al, Canada (which is further north than New Zealand or MIT).

Additionally, a recent paper \todo{PICK SOTA!}

\subsection{Outline}

foo

\section{The Fourier transform}

foo

\subsection{Fourier transform}

foo

\subsection{Discrete Fourier transform}

foo

\subsection{Fast Fourier transform}

foo

\subsection{Common operations with the Fourier transform}

correlations, convolutions, STFT

\subsubsection{Correlation and convolution}

\subsubsection{Time-frequency analysis}

\subsubsection{Cepstral coefficients}

\subsection{Usage in music systems}

foo

\section{Variants of the Fourier transform}

foo

\subsection{Psychoacoustic DFT}

\qquad \textit{abbreviations:} NFT (nonuniform Fourier transform), NUFFT (nonuniform FFT), WDFT (warped DFT)

A characteristic of the standard DFT is that the output frequencies are evenly spaced. For an $N$-point DFT, the DFT coefficient at bin (or index) 0 corresponds to the DC component of the sound, and every non-zero coefficient corresponds to a frequency $\frac{n}{\mathit{fs}}, n \in (0, N)$, where $\mathit{fs}$ is the sampling rate of the input signal. In practice, since frequencies above the Nyquist rate cannot be represented, the useful bins of a DFT are between $1$ and $\frac{N}{2}+1$, representing the frequencies $\frac{1}{\mathit{fs}}$ Hz and $\frac{\mathit{fs}}{2} = f_{\text{nyq}}$ Hz. Each bin in between is spaced $\frac{1}{\mathit{fs}}$ apart from the previous and next bin.

The human auditory system has a non-linear frequency response, with a finer frequency resolution at lower frequencies. \todo{BCJ Moore or Plack} This has implications in both speech and music processing. In music, low frequency bass notes establish the harmonic basis of a song. \todo{doerfler} Common approaches, such as spectrograms using the log, Mel, Bark, or octave psychoacoustic frequency scales, involve reducing an $N$-point DFT to an $L$-point DFT where $L$ represents nonuniform frequencies on the nonlinear Mel or Bark scales \todo{cite psychoacoustic scales}.

\todo{nice psychoacoustic diagrams here}

\subsubsection{Warped DFT}

The warped DFT \cite{warped1, warped2}.

 In warped DFT, such nonuni-form frequency samples are obtained by a nonlinear warping ofuniform spacing. In nonuniform DFT, the frequency samples arecompletely arbitrary [3]. The above DFT variants are arrangedin the increasing order of flexibility.1


\subsubsection{Nonuniform DFT}

A generalization of the warped DFT.

The variant of the DFT which attempts to output $L$ non-linear frequencies directly (without first computing linearly-spaced $N > L$ points, many of which will be discarded) is the nonuniform (or nonequispaced) DFT, also referred to as the NFT (nonuniform or nonequispaced Fourier transform) and NUFFT (nonuniform Fast Fourier transform). The most recent implementation of the NUFFT is by \citet{nufft1} with source code available,\footnote{\href{https://finufft.readthedocs.io/en/latest/index.html}{https://finufft.readthedocs.io/en/latest/index.html}} which in turn derives from earlier work \cite{nufft2, nufft3}.

\subsubsection{Relation to the Constant-Q transform}

foo

\subsection{Sparse and pruned DFT}


A statement on pruned FFTs from the fftw website\footnote{\href{http://www.fftw.org/pruned.html}{http://www.fftw.org/pruned.html}} recommends ``not bothering to consider a pruned 1d FFT,'' since the computational cost savings of pruning may not be worth losing the optimization of the regular FFT for the 1-dimensional case (it is a different story for 2d or higher dimension Fourier transforms, but those are not used for music analysis). This thesis also seeks to revisit this idea, to see whether sparse or pruned FFTs for 1-dimensional audio signals are worth using. In machine or deep learning systems, thousands and thousands of FFTs are computed for large amounts of audio training data, and my hypothesis is that seemingly negligible computational savings for small 1d FFTs might stack up when performing many of them.

Hassanieh et al say that k now only has to be a little less than n, not very less - \todo{tie into fftw}

 could lead to even better results, such as the sparse or pruned FFTs which compute a $k$-point FFT in $O(k \log n)$ time where $k << n$, assuming that those $k$ frequency components are more important than the full $n$ outputs of the regular FFT.



\subsection{Discrete Fractional Fourier transform}

foo

\subsubsection{Sparse Discrete Fractional Fourier Transform}

foo

\section{Fourier transform libraries}

foo

\subsection{Existing libraries}

foo

\subsection{Designing a new library}

foo

\subsection{Performance optimizations}

foo

\subsection{Benchmarks}

foo

\section{Results}

foo

\subsection{Outputs of implemented variants}

foo

\subsection{Memory footprint}

foo

\subsection{Computational footprint}

foo

\subsection{Substitution in a real-world music system}

foo

\section{Conclusions}

foo

\subsection{Evaluation}

foo

\subsection{Outlook}

foo

\subsection{Summary}

foo

\vfill
\clearpage % force a page break before references

%\nocite{*}
\section{References}
\printbibliography[heading=none]

\vfill
\clearpage %force a page break

\begin{appendices}

\section{Extra stuff}
\label{appendix:rthpss}

\end{appendices}

\end{document}
