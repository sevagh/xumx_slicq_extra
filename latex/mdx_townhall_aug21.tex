\documentclass[usenames,dvipsnames]{beamer}
\usetheme{Boadilla}
\usepackage{hyperref}
\usepackage{graphicx}
\usepackage{multimedia}
\usepackage{fancyvrb}
\usepackage{soul}
\usepackage{multicol}
\usepackage{optparams}
\usepackage{adjustbox}
\usepackage{tikz}
\usetikzlibrary{shapes,positioning}
\newcommand{\foo}{\hspace{-2.3pt}$\bullet$ \hspace{5pt}}
\usepackage{subfig}
\usepackage[
    backend=biber,
    natbib=true,
    style=numeric,
    sorting=none,
    style=verbose-ibid,
    maxcitenames=1, %remove this outside of toy presentations
]{biblatex}
\addbibresource{citations.bib}
\usepackage{pgfpages}
\usepackage{xcolor}
\definecolor{ao(english)}{rgb}{0.0, 0.5, 0.0}
\definecolor{burgundy}{rgb}{0.5, 0.0, 0.13}
%\setbeameroption{show notes}
%\setbeameroption{show notes on second screen=right}
\setbeameroption{hide notes}
\newcommand\ThesisTitle{Music Source Separation with the sliCQ Transform}


\title{xumx-sliCQ}
\subtitle{sevagh's submission to the Music Demixing Challenge}
\author{Sevag H.}
\date{August 21, 2021}
\setbeamertemplate{navigation symbols}{}

\AtEveryBibitem{%
  \clearfield{pages}%
  \clearfield{volume}%
  \clearfield{number}%
  \clearlist{journal}%
  \clearfield{booktitle}%
}

\renewbibmacro{in:}{}

\AtEveryCitekey{%
  \clearfield{pages}%
  \clearfield{volume}%
  \clearfield{number}%
  \clearfield{doi}%
  \clearfield{journal}%
  \clearlist{journal}%
  \clearfield{booktitle}%
  \clearfield{isbn}%
  \clearfield{title}%
  \clearfield{url}%
\ifentrytype{article}{
    \clearfield{journal}%
}{}
\ifentrytype{inproceedings}{
    \clearfield{booktitle}%
}{}
}

\begin{document}

\begin{frame}
\maketitle
\end{frame}

\begin{frame}
	\frametitle{Who am I}
	\begin{enumerate}
		\item
			Sevag H. (sevagh, my screen name, = first name + last initial)
		\item
			Electrical engineering background, Linux infrastructure engineer at Pandora (U.S. music streaming service)
		\item
			Active open-source enthusiast on GitHub: \href{https://github.com/sevagh}{https://github.com/sevagh}
			I write music-related code because I like to code and I like music
		\item
			Current student in the Master of Arts, Music Technology program at McGill -- member of the Distributed Digital Music Archives \& Libraries lab: \href{https://ddmal.music.mcgill.ca/}{https://ddmal.music.mcgill.ca/} led by Prof. Ichiro Fujinaga
	\end{enumerate}
	Why did I choose to participate in the MDX challenge?
	\begin{enumerate}
		\item
			I first looked at the median-filtering harmonic/percussive source separation\footfullcite{fitzgerald1} algorithm as preprocessing for joint pitch + beat tracking
		\item
			Fascinated by music demixing since, from the signal processing perspective of time-frequency resolution and realtime performance
		\item
			Timing of competition coincided with tentative thesis -- replace the STFT in Open-Unmix with a different time-frequency transform
	\end{enumerate}
\end{frame}

\begin{frame}
	\frametitle{Improved HPSS}
	\begin{enumerate}
	\item
		Multi-STFT HPSS.\footfullcite{driedger} In spectral demixing/source separation, window size of STFT matters, one size does not fit all\footfullcite{tftradeoff1, tftradeoff2}
	\item
		From musical and auditory aspects, frequency resolution should increase from high to low frequencies (vice-versa for time resolution)\footfullcite{cqtransient}
	\item
		Music must be analyzed with long windows/$\uparrow \Delta f$ in the low frequencies for the harmonic basis of the bass notes, and short windows/$\uparrow \Delta t$ in the high frequency region for detecting transients\footfullcite{doerflerphd}
	\item
		CQT HPSS.\footfullcite{fitzgerald2} Constant-Q transform\footcite{jbrown} analyzes music on the 12-tone Western pitch scale; long windows in low frequencies, and short windows in high frequencies. Approximate inverse\footfullcite{klapuricqt} in librosa
	\end{enumerate}
\end{frame}

\begin{frame}
	\frametitle{CQT vs. STFT spectrograms}
	\begin{figure}[ht]
		\centering
		\subfloat[CQT, 12 bins/octave]{\includegraphics[height=3.3cm]{./images-gspi/glock_cqt12.png}}
		\subfloat[CQT, 24 bins/octave]{\includegraphics[height=3.3cm]{./images-gspi/glock_cqt24.png}}
		\subfloat[CQT, 48 bins/octave]{\includegraphics[height=3.3cm]{./images-gspi/glock_cqt48.png}}\\
		\vspace{-0.5em}
		\subfloat[STFT, window = 256]{\includegraphics[height=3.3cm]{./images-gspi/glock_stft256.png}}
		\subfloat[STFT, window = 1024]{\includegraphics[height=3.3cm]{./images-gspi/glock_stft1024.png}}
		\subfloat[STFT, window = 4096]{\includegraphics[height=3.3cm]{./images-gspi/glock_stft4096.png}}
	\end{figure}
\end{frame}

\begin{frame}
	\frametitle{My approach part 1 -- sliCQ Transform}
	\begin{enumerate}
	\item
		Gabor\footcite{gabor1946} wrote the first published use of the short-time Fourier transform.\footcite{dictionary} Nonstationary Gabor transform\footcite{balazs} and the realtime sliCQ transform\footcite{invertiblecqt, slicq, variableq1} are built from frame theory, a mathematical technique for finding ``redundant, stable way[s] of representing a signal''\footcite{framesintro} -- STFT-like operation but with windows that vary with time
	\item
		CQT is the motivation for the NSGT/sliCQT, but they can use any monotonically increasing frequency scale (log/cq, mel, Bark, etc.), and output the familiar Fourier coefficients with perfect inverse
	\item
		Goal: use the sliCQT to choose optimal time-frequency resolutions per frequency and boost spectral models past waveform models\footfullcite{demucs}
	\end{enumerate}
\end{frame}

\begin{frame}[fragile]
	\frametitle{sliCQT configuration for xumx-sliCQ}
	\begin{figure}[ht]
		\centering
		\includegraphics[height=3cm]{./images-gspi/gspi_xumx_slicq_params.png}
		\caption{sliCQT spectrogram of glockenspiel}
	\end{figure}
	PyTorch/GPU-capable fork: \url{https://github.com/sevagh/nsgt}
	\begin{verbatim}
	$ python examples/spectrogram.py \
		 --scale bark --bins 262 --fmin 32.9 \
		 --fmax 22050 --sllen 18060 --trlen 4514  --plot ./gspi.wav
	\end{verbatim}
	Settings were chosen from my parameter search for xumx-sliCQ; read more here: \url{https://github.com/sevagh/xumx-sliCQ/blob/main/docs/slicq_params.md}
\end{frame}

\begin{frame}
	\frametitle{My approach part 2 -- xumx-sliCQ, a variant of Open-Unmix}
	\begin{itemize}
		\item
			xumx-sliCQ: \url{https://github.com/sevagh/xumx-sliCQ}
		\item
			Even if not using the Bi-LSTM\footcite{umx} of Open-Unmix, using the template \url{https://github.com/sigsep/open-unmix-pytorch} gives you a full ecosystem for spectral demixing including MUSDB18-HQ loader, BSS evaluator, and a post-processing Wiener-EM step
		\item
			Replaces the STFT with sliCQT, and uses convolutional architectures\footfullcite{plumbley1, plumbley2} instead of Bi-LSTM (time-frequency 2D convolutional filters, applicable to any spectral transform)
		\item
			Adds CrossNet-Open-Unmix (aka X-UMX) features\footcite{xumx} including combining all 4 targets with multi-target and cross-domain (time and frequency) loss
	\end{itemize}
\end{frame}

\begin{frame}
	\frametitle{What worked vs. didn't work}
	\begin{enumerate}
		\item
			sliCQT has a matrix form with zero-padding
		\item
			Bad neural network convergence and demixing results from the matrix form -- different frequency bins have a different temporal frame rate
		\item
			Switched to using the ragged form, and applying convolutional layers to grouped frequency bins
	\end{enumerate}
	\begin{figure}[ht]
		\centering
		\includegraphics[width=10cm]{./images-blockdiagrams/slicq_shape.png}\\
		\caption{sliCQT ragged shape for one slice of the time-domain waveform}
	\end{figure}
\end{frame}

\begin{frame}
	\frametitle{Competition results}
	\begin{enumerate}
	\item
		Luck-based and undisciplined approach with random network architectures -- consider it a ``proof-of-concept'' of sliCQT demixing
	\item
		Invitation to demixing researchers: more rigorous, data-driven approaches to the sliCQT parameter search and network architecture might produce a good result
	\end{enumerate}
	\begin{figure}[ht]
		\centering
		\includegraphics[width=8cm]{./images-misc/leaderboard_header.png}\\
		\vspace{-0.25em}
		\includegraphics[width=8cm]{./images-misc/leaderboard_myplace.png}
		\caption{Leaderboard A position of xumx-sliCQ}
		\vspace{-1em}
	\end{figure}

\end{frame}

\begin{frame}
	\frametitle{My impressions on the competition}
	\begin{enumerate}
	\item
	 ``It is definitely a great environment to push the limit, had it been for a paper, I would have stopped sooner.'' -- defossez
	 \item
		 There was a lot of active discussion on the board (I got help wherever I needed it) and everything felt set up for participants to succeed
	 \item
		 I could have paid more attention to what my peers were doing, or used a different baseline, but after the first Town Hall I decided I want to see how far I could go with the sliCQT + UMX
	 \item
		 GitLab submission process worked better than I expected, submissions were easy, and the containers had important deep learning Python libraries already installed -- minimal conda/pip/Python issues
	 \item
		 I got the impression that success was more about data than clever coding. I should improve my machine learning skills for the 2022 competition
	\end{enumerate}
\end{frame}

\end{document}
