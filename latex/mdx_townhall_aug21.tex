\documentclass[usenames,dvipsnames]{beamer}
\usetheme{Boadilla}
\usepackage{hyperref}
\usepackage{graphicx}
\usepackage{multimedia}
\usepackage{fancyvrb}
\usepackage{soul}
\usepackage{multicol}
\usepackage{optparams}
\usepackage{adjustbox}
\usepackage{tikz}
\usetikzlibrary{shapes,positioning}
\newcommand{\foo}{\hspace{-2.3pt}$\bullet$ \hspace{5pt}}
\usepackage{subfig}
\usepackage[
    backend=biber,
    natbib=true,
    style=numeric,
    sorting=none,
    style=verbose-ibid,
    maxcitenames=1, %remove this outside of toy presentations
]{biblatex}
\addbibresource{citations.bib}
\usepackage{pgfpages}
\usepackage{xcolor}
\definecolor{ao(english)}{rgb}{0.0, 0.5, 0.0}
\definecolor{burgundy}{rgb}{0.5, 0.0, 0.13}
%\setbeameroption{show notes}
%\setbeameroption{show notes on second screen=right}
\setbeameroption{hide notes}
\newcommand\ThesisTitle{Music Source Separation with the sliCQ Transform}


\title{xumx-sliCQ}
\subtitle{sevagh's submission to the Music Demixing Challenge}
\author{Sevag H.}
\date{August 21, 2021}
\setbeamertemplate{navigation symbols}{}

\AtEveryBibitem{%
  \clearfield{pages}%
  \clearfield{volume}%
  \clearfield{number}%
  \clearlist{journal}%
  \clearfield{booktitle}%
}

\renewbibmacro{in:}{}

\AtEveryCitekey{%
  \clearfield{pages}%
  \clearfield{volume}%
  \clearfield{number}%
  \clearfield{doi}%
  \clearfield{journal}%
  \clearlist{journal}%
  \clearfield{booktitle}%
  \clearfield{isbn}%
  \clearfield{title}%
  \clearfield{url}%
\ifentrytype{article}{
    \clearfield{journal}%
}{}
\ifentrytype{inproceedings}{
    \clearfield{booktitle}%
}{}
}

\begin{document}

\begin{frame}
\maketitle
\end{frame}

\begin{frame}
	\frametitle{Who am I}
	\begin{enumerate}
		\item
			Electrical engineering background
		\item
			Linux engineer at Pandora (U.S. music streaming service)
		\item
			Active open-source enthusiast on GitHub: \href{https://github.com/sevagh}{https://github.com/sevagh}
		\item
			Student in the Master of Arts, Music Tech program at McGill -- member of the Distributed Digital Music Archives \& Libraries lab led by Prof. Ichiro Fujinaga
	\end{enumerate}
\end{frame}

\begin{frame}
	\frametitle{Why did I participate?}
	Started with median-filtering harmonic/percussive source separation (HPSS)\footfullcite{fitzgerald1} algorithm as preprocessing for joint pitch + beat tracking. Fascinated by music demixing since.
	\begin{figure}[ht]
		\centering
		\vspace{-0.5em}
		\includegraphics[height=6cm]{./images-misc/hpss.png}
	\end{figure}
\end{frame}

\begin{frame}
	\frametitle{Improving harmonic/percussive source separation (HPSS)}
	\begin{enumerate}
	\item
		Short-time Fourier transform (STFT) window size matters per-target\footfullcite{tftradeoff1, tftradeoff2}
	\item
		From musical and auditory aspects, frequency resolution should increase from high to low frequencies (vice-versa for time resolution)\footfullcite{cqtransient}
	\item
		Use long windows/$\uparrow \Delta f$ in low frequencies, and short windows/$\uparrow \Delta t$ in high frequencies to analyze music (harmonic basis and transients)\footfullcite{doerflerphd}
	\item
		window=4096 for harmonic, window=256 for percussive in HPSS\footfullcite{driedger}
	\item
		Constant-Q Transform (CQT) and multiple STFTs in HPSS\footfullcite{fitzgerald2}
	\item
		CQT\footcite{jbrown, klapuricqt} uses long windows in low frequencies and short windows in high frequencies for the 12-tone Western pitch scale
	\end{enumerate}
\end{frame}

\begin{frame}
	\frametitle{Short-time Fourier Transform vs. Constant-Q Transform}
	\begin{figure}[ht]
		\centering
		\vspace{-1em}
		\subfloat[STFT, window = 256]{\includegraphics[height=3.3cm]{./images-gspi/glock_stft256.png}}
		\subfloat[STFT, window = 1024]{\includegraphics[height=3.3cm]{./images-gspi/glock_stft1024.png}}
		\subfloat[STFT, window = 4096]{\includegraphics[height=3.3cm]{./images-gspi/glock_stft4096.png}}
		\vspace{-0.5em}
		\subfloat[CQT, 12 bins/octave]{\includegraphics[height=3.3cm]{./images-gspi/glock_cqt12.png}}
		\subfloat[CQT, 24 bins/octave]{\includegraphics[height=3.3cm]{./images-gspi/glock_cqt24.png}}
		\subfloat[CQT, 48 bins/octave]{\includegraphics[height=3.3cm]{./images-gspi/glock_cqt48.png}}
		\vspace{-0.5em}
	\end{figure}
\end{frame}

\note{
	\begin{itemize}
		\item
			small window STFT has blurry frequency bins, sharp temporal events
		\item
			long window STFT loses some of the frequency components of the glockenspiel
		\item
			CQT has sharp temporal events and more frequency contents for any bins-per-octave
	\end{itemize}
}

\begin{frame}
	\frametitle{My approach -- sliCQ}
	\begin{enumerate}
	\item
		Nonstationary Gabor Transform (NSGT)\footcite{balazs}, realtime sliCQ Transform\footcite{invertiblecqt, slicq, variableq1}
	\item
		STFT-like transforms with windows that vary with time
	\item
		CQT motivates the NSGT/sliCQ, but can use any monotonically increasing frequency scale (log/cq, mel, Bark, etc.)
	\item
		Outputs the familiar Fourier coefficients with perfect inverse
	\item
		Competition goal: use sliCQ in Open-Unmix (UMX)
	\end{enumerate}
	\begin{figure}[ht]
		\centering
		\vspace{-0.5em}
		\includegraphics[height=2.5cm]{./images-gspi/gspi_xumx_slicq_params.png}
		\vspace{-0.5em}
		\caption{sliCQ: 262-bin Bark scale, 32.9--22050 Hz}
		\vspace{-0.5em}
	\end{figure}
\end{frame}
\note{
	\begin{itemize}
		\item
			Optimal time-frequency resolution per frequency bin might improve results
		\item
			Bridge the gap between spectral models and waveform models (by improving their time-frequency resolution)
	\end{itemize}
}

\begin{frame}
	\frametitle{My approach -- xumx-sliCQ}
	\begin{enumerate}
	\item
		xumx-sliCQ: \url{https://github.com/sevagh/xumx-sliCQ}
	\item
		PyTorch fork of NSGT/sliCQ: \url{https://github.com/sevagh/nsgt}
	\item
		Uses UMX\footcite{umx} PyTorch template + CrossNet-Open-Unmix (X-UMX)\footcite{xumx}
	\item
		Replace STFT with sliCQT, replace Bi-LSTM with convolutions\footfullcite{plumbley2}
\end{enumerate}
	\begin{figure}[ht]
		\centering
		\includegraphics[height=3.5cm]{./images-blockdiagrams/xumx_slicq_system_compressed.png}
		\vspace{-1em}
	\end{figure}
\end{frame}

\begin{frame}
	\frametitle{What worked vs. didn't work}
	sliCQT has a matrix form with zero-padding; poor neural network convergence; different frequency bins = different temporal frame rate\\
	Use ragged form, write different conv layers for each time-frequency resolution block:
	\begin{figure}[ht]
		\centering
		\includegraphics[height=5cm]{./images-blockdiagrams/slicq_shape.png}
	\end{figure}
\end{frame}

\begin{frame}
	\frametitle{Competition results}
	\begin{enumerate}
	\item
		Luck-based approach with network copied from STFT models
	\item
		Invitation to demixing researchers: more rigorous, data-driven approaches to the sliCQT parameter search and network architectures
	\end{enumerate}
	\begin{figure}[ht]
		\centering
		\includegraphics[width=8cm]{./images-misc/leaderboard_header.png}\\
		\vspace{-0.25em}
		\includegraphics[width=8cm]{./images-misc/leaderboard_myplace.png}
		\caption{Leaderboard A position of xumx-sliCQ}
		\vspace{-1em}
	\end{figure}

\end{frame}

\begin{frame}
	\frametitle{My impressions on the competition}
	\begin{enumerate}
	\item
	 ``It is definitely a great environment to push the limit, had it been for a paper, I would have stopped sooner.'' -- defossez
	 \item
		 There was a lot of active discussion on the board and everything felt set up for participants to succeed
	 \item
		 GitLab submission process worked well, submissions were easy, and the containers had important Python libraries already installed
	 \item
		 I look forward to the 2022 competition
	\end{enumerate}
\end{frame}

\end{document}
