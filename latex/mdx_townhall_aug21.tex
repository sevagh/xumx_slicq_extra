\documentclass[usenames,dvipsnames]{beamer}
\usetheme{Boadilla}
\usepackage{hyperref}
\usepackage{graphicx}
\usepackage{multimedia}
\usepackage{fancyvrb}
\usepackage{soul}
\usepackage{multicol}
\usepackage{optparams}
\usepackage{adjustbox}
\usepackage{tikz}
\usetikzlibrary{shapes,positioning}
\newcommand{\foo}{\hspace{-2.3pt}$\bullet$ \hspace{5pt}}
\usepackage{subfig}
\usepackage[
    backend=biber,
    natbib=true,
    style=numeric,
    sorting=none,
    style=verbose-ibid,
    maxcitenames=1, %remove this outside of toy presentations
]{biblatex}
\addbibresource{citations.bib}
\usepackage{pgfpages}
\usepackage{xcolor}
\definecolor{ao(english)}{rgb}{0.0, 0.5, 0.0}
\definecolor{burgundy}{rgb}{0.5, 0.0, 0.13}
%\setbeameroption{show notes}
%\setbeameroption{show notes on second screen=right}
\setbeameroption{hide notes}
\newcommand\ThesisTitle{Music Source Separation with the sliCQ Transform}


\title{xumx-sliCQ}
\subtitle{sevagh's submission to the Music Demixing Challenge}
\author{Sevag H.}
\date{August 21, 2021}
\setbeamertemplate{navigation symbols}{}

\AtEveryBibitem{%
  \clearfield{pages}%
  \clearfield{volume}%
  \clearfield{number}%
  \clearlist{journal}%
  \clearfield{booktitle}%
}

\renewbibmacro{in:}{}

\AtEveryCitekey{%
  \clearfield{pages}%
  \clearfield{volume}%
  \clearfield{number}%
  \clearfield{doi}%
  \clearfield{journal}%
  \clearlist{journal}%
  \clearfield{booktitle}%
  \clearfield{isbn}%
  \clearfield{title}%
  \clearfield{url}%
\ifentrytype{article}{
    \clearfield{journal}%
}{}
\ifentrytype{inproceedings}{
    \clearfield{booktitle}%
}{}
}

\begin{document}

\begin{frame}
\maketitle
\end{frame}

\begin{frame}
	\frametitle{Who am I}
	\begin{enumerate}
		\item
			Sevag H. (sevagh, my screen name, = first name + last initial)
		\item
			Electrical engineering background (B.Eng from McGill in 2014) -- first intro to signal processing
		\item
			Linux infrastructure engineer at Pandora (U.S. music streaming)
		\item
			Active open-source enthusiast on GitHub: \href{https://github.com/sevagh}{https://github.com/sevagh}
			I write music-related code because I like music, and I like to practice coding in a domain that I personally care about, to stay motivated
		\item
			Current student in the Master of Arts, Music Technology program at McGill -- member of the Distributed Digital Music Archives \& Libraries lab: \href{https://ddmal.music.mcgill.ca/}{https://ddmal.music.mcgill.ca/} led by Prof. Ichiro Fujinaga. I chose grad school to have some guidance after hitting a limit in my self-learning
	\end{enumerate}
\end{frame}

\begin{frame}
	\frametitle{MDX challenge motivations}
	Why did I choose to participate?
	\begin{enumerate}
		\item
			I first looked at the median-filtering harmonic/percussive source separation\footfullcite{fitzgerald1} algorithm as preprocessing for joint pitch + beat tracking
		\item
			Since then, spent 1+ year studying music demixing (mostly from the perspective of time-frequency resolution)
		\item
			Timing of this competition coincided with my tentative thesis topic -- replace the STFT in Open-Unmix with a different time-frequency transform, and see if we can improve the results
	\end{enumerate}
\end{frame}

\begin{frame}
	\frametitle{Improved demixing with multiple STFTs}
	\begin{figure}[ht]
		\vspace{-1em}
		\centering
		\subfloat[128-sample window]{\includegraphics[height=2.75cm]{./images-gspi/gspi_hamm_128.png}}
		\hspace{0.5em}
		\subfloat[16384-sample window]{\includegraphics[height=2.75cm]{./images-gspi/gspi_hamm_16384.png}}\\
		\caption{STFT spectrograms of glockenspiel}
		\vspace{-1em}
	\end{figure}
	Music has conflicting requirements in the STFT; it needs long windows in the low frequency region for a high frequency resolution, because bass notes lay the harmonic basis, and short windows in the high frequency region for transients (timbre and rhythm)\footfullcite{doerflerphd}\\
	Multi-STFT HPSS\footfullcite{driedger}, different-sized time-frequency filters on one STFT\footfullcite{plumbley2}
\end{frame}

\begin{frame}
	\frametitle{Improved HPSS with the constant-Q transform}
	\begin{figure}[ht]
		\vspace{-1em}
		\centering
		\subfloat[12 bins-per-octave]{\includegraphics[height=2.75cm]{./images-gspi/glock_cqt12.png}}
		\hspace{1em}
		\subfloat[48 bins-per-octave]{\includegraphics[height=2.75cm]{./images-gspi/glock_cqt48.png}}
		\caption{CQT spectrograms of glockenspiel}
		\vspace{-1em}
	\end{figure}
	The Constant-Q transform\footcite{jbrown} was designed to analyze music on the 12-tone Western pitch scale, using long windows in low frequency regions and short windows in high frequency regions, with approximate inverse\footfullcite{klapuricqt}\\
	CQT HPSS\footfullcite{fitzgerald2}, CQT music demixing\footfullcite{bettermusicsep}
\end{frame}

\begin{frame}
	\frametitle{My approach -- the sliCQ Transform}
	\begin{enumerate}
	\item
		Nonstationary Gabor Transform\footcite{balazs} and the realtime variant the sliCQ transform\footcite{invertiblecqt} are time-frequency transforms built from frame theory, which is a mathematical technique for computing ``redundant, stable way[s] of representing a signal''\footfullcite{framesintro}
	\item
		The CQT is the motivating application for the NSGT and sliCQT. However, they can be used to implement a perfectly invertible TFR with any (monotonically increasing) frequency scale, by applying an STFT with a window size that varies with time. It uses the familiar Fourier coefficients, and supports constant-Q/Western pitch scale, mel, Bark, ERBlets,\footfullcite{variableq1} etc.
	\item
		We can define any frequency scale suitable for the application or nature of the input signal
	\end{enumerate}
\end{frame}

\begin{frame}
	\frametitle{What didn't work -- the matrix form of the sliCQT}
\end{frame}

\begin{frame}
	\frametitle{My impressions on the competition}
 ``It is definitely a great environment to push the limit, had it been for a paper, I would have stopped sooner.'' -- defossez
\end{frame}

\end{document}
