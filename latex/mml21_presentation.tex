\documentclass[usenames,dvipsnames]{beamer}
\usetheme{Boadilla}
\usepackage{hyperref}
\usepackage{graphicx}
\usepackage{multimedia}
\usepackage{fancyvrb}
\usepackage{soul}
\usepackage{multicol}
\usepackage{optparams}
\usepackage{adjustbox}
\usepackage{tikz}
\usetikzlibrary{shapes,positioning}
\newcommand{\foo}{\hspace{-2.3pt}$\bullet$ \hspace{5pt}}
\usepackage{subfig}
\usepackage[
    backend=biber,
    natbib=true,
    style=numeric,
    sorting=none,
    style=verbose-ibid,
    maxcitenames=1, %remove this outside of toy presentations
]{biblatex}
\addbibresource{citations_presentation.bib}
\usepackage{pgfpages}
\usepackage{xcolor}
\definecolor{ao(english)}{rgb}{0.0, 0.5, 0.0}
\definecolor{burgundy}{rgb}{0.5, 0.0, 0.13}
%\setbeameroption{show notes}
%\setbeameroption{show notes on second screen=right}
\setbeameroption{hide notes}
\newcommand\ThesisTitle{Music Source Separation with the sliCQ Transform}


\title{Music demixing with the sliCQ transform}
\author{Sevag Hanssian}
\date{December 15, 2021}
%\date{}
\setbeamertemplate{navigation symbols}{}

\AtEveryBibitem{%
  \clearfield{pages}%
  \clearfield{school}%
  \clearfield{volume}%
  \clearfield{number}%
  \clearlist{journal}%
  \clearfield{journal}%
  \clearfield{booktitle}%
}

\renewbibmacro{in:}{}

\AtEveryCitekey{%
  \clearfield{pages}%
  \clearfield{volume}%
  \clearfield{school}%
  \clearfield{number}%
  \clearfield{doi}%
  \clearfield{journal}%
  \clearlist{journal}%
  \clearfield{booktitle}%
  \clearfield{isbn}%
  \clearfield{title}%
  \clearfield{url}%
\ifentrytype{article}{
    \clearfield{journal}%
}{}
\ifentrytype{inproceedings}{
    \clearfield{booktitle}%
}{}
\ifentrytype{article}{
    \clearfield{journal}%
}{}
\ifentrytype{phdthesis}{
    \clearfield{school}%
}{}
}

\begin{document}

\begin{frame}
\maketitle
\end{frame}

\begin{frame}
	\frametitle{Music source separation, demixing, and unmixing}
	Music source separation: extract an estimate of an isolated source (or target) from mixed musical audio (e.g., harmonic/percussive, vocals, drums, bass, piano)\\
	Music demixing (or unmixing): estimate multiple sources (vocals, drums, bass, other\footcite{musdb18hq}) that can be summed back to the original mix. Multiple MSS subproblems, reversing the linear mixing of stems in the recording studio (stem datasets can be used for mixing and demixing)
	\begin{figure}[ht]
		\centering
		\vspace{-0.25em}
		\includegraphics[width=7.5cm]{./images-mss/mixdemix.png}
		\vspace{-0.75em}
	\end{figure}
	Popular approach: musical source models, which are ``model-based approaches that attempt to capture the spectral characteristics of the target source'' \footcite[36]{musicsepgood} with \textbf{time-frequency masks}
\end{frame}

\begin{frame}
	\frametitle{Music source separation with DNN and spectrograms}
	\begin{figure}
		\centering
		\vspace{-1.25em}
		\subfloat{\includegraphics[width=4.25cm]{./images-mss/mss1.png}}
		\subfloat{\includegraphics[width=4.5cm]{./images-mss/mss2.png}}\\
		\vspace{-1em}
		\subfloat{\includegraphics[width=8.75cm]{./images-mss/mask_simple.png}}\\
		\vspace{-0.75em}
		\subfloat{\includegraphics[width=7.5cm]{./images-mss/mssdnn.png}}
	\end{figure}
\end{frame}

\begin{frame}
	\frametitle{Phase performance ceiling}
	\begin{enumerate}
		\item
			Simplifying assumption: estimate magnitude spectrograms, use the phase of the original mixed audio. Called ``noisy phase'' \footcite{noisyphase1}. Done by Open-Unmix (UMX), CrossNet-Open-Unmix (X-UMX) \footcite{umx, xumx}, and many other popular \& near-SOTA models
		\item
			Why? Phase is hard to model!\footnote{\url{https://source-separation.github.io/tutorial/basics/phase.html\#why-we-don-t-model-phase}}
	\end{enumerate}
	\begin{figure}[ht]
		\centering
		\includegraphics[width=7.5cm]{./images-mss/whynophase.png}
	\end{figure}
\end{frame}

\begin{frame}
	\frametitle{Time-frequency tradeoff in the STFT and MDX}
	Joint time-frequency analysis is important for signals whose frequencies change with time.\footcite{gabor1946} Take the Fourier transform of local windows of the signal, i.e., the STFT. Change window size to trade off time and frequency:
	\begin{figure}[ht]
		\centering
		\vspace{-1.5em}
		\subfloat{\includegraphics[height=2cm]{./images-tftheory/gabor3.png} \includegraphics[height=1.75cm]{./images-tftheory/gabor4.png}}\\
		\vspace{-1em}
		\subfloat{\includegraphics[height=2.7cm]{./images-mml-presentation/glock_stft_1024.png}}
		\subfloat{\includegraphics[height=2.7cm]{./images-mml-presentation/glock_stft_4096.png}}
		\subfloat{\includegraphics[height=2.7cm]{./images-mml-presentation/glock_stft_256.png}}
		\vspace{-1.25em}
	\end{figure}
	In music source separation, window size matters per-target.\footcite{tftradeoff1} Short-window for percussion, long-window for harmonic
\end{frame}

\begin{frame}
	\frametitle{CQT, NSGT, sliCQT}
	\begin{enumerate}
	\item
		For musical and auditory reasons, we want high frequency resolution at low frequencies and high time resolution at high frequencies \footcite{doerflerphd}
	\item
		CQT\footcite{jbrown} uses long windows in low frequencies and short windows in high frequencies for the 12-tone Western pitch scale
	\item
		Nonstationary Gabor Transform (NSGT) and sliCQT\footcite{balazs, slicq} are TF transforms with Fourier coefficients, perfect inverse, and varying windows to create a varying time-frequency resolution
	\item
		sliCQT params chosen for max quality of the noisy-phase waveform
	\end{enumerate}
	\begin{figure}[ht]
		\centering
		\vspace{-0.5em}
		\includegraphics[height=2.7cm]{./images-mml-presentation/spectrograms_comparison.png}
	\end{figure}
\end{frame}

\begin{frame}
	\frametitle{xumx-sliCQ}
	\begin{enumerate}
		\item
			My goal: improve Open-Unmix by replacing STFT with sliCQT
		\item
			My model submitted to the MDX21 challenge and workshop: \url{https://github.com/sevagh/xumx-sliCQ}
		\item
			Use Convolutional Denoising Autoencoder\footcite{plumbley1, plumbley2} neural architecture
		\item
			Scored 3.6 dB vs. 4.6 dB (UMX) and 5.54 dB (X-UMX); there is still room for improvement
	\end{enumerate}
	\begin{figure}[ht]
		\centering
		\vspace{-1.15em}
		\subfloat{\includegraphics[width=8cm]{./images-blockdiagrams/generic_mdx.png}}\\
		\vspace{-0.5em}
		\subfloat{\includegraphics[width=8cm]{./images-blockdiagrams/umx_clean.png}}\\
		\vspace{-0.5em}
		\subfloat{\includegraphics[width=8cm]{./images-blockdiagrams/xumx_slicq_clean.png}}
	\end{figure}
\end{frame}

\begin{frame}
	\frametitle{MDX 21 winners and current trends}
	\begin{enumerate}
		\item
			Previously, music demixing systems were submitted to and evaluated at SiSEC (Signal Separation Evaluation Campaign). This year: MDX (Music Demixing Challenge) ISMIR 2021 @ AICrowd, follow-up MDX21 workshop, satellite @ ISMIR 2021
		\item
			\textbf{ISMIR 2021}: Model that uses the complex spectrogram (i.e. includes phase) and uses complex masks\footcite{kong2021decoupling}
		\item
			\textbf{MDX21}: 1: Demucs++\footcite{demucsplus} (waveforms + complex spectrogram), 2: KUIELAB-MDX-Net \footcite{choi2021} (waveforms + magnitude spectrogram), 3: Danna-Sep\footcite{dannasep} (waveform + magnitude spectrogram, use complex spectrogram in loss function)
	\end{enumerate}
	Properties in common: blending networks, waveforms (implicitly includes phase), complex spectrograms/masks, mixing spectrogram and waveform models
\end{frame}

\end{document}
