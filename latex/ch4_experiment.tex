\documentclass[report.tex]{subfiles}
\begin{document}

\section{Experiment}

\subsection{sliCQ transform performance analysis}

\subsection{sliCQ transform parameter search}

\subsection{Neural network training}

The training parameters of xumx-sliCQ were kept similar to Open-Unmix and CrossNet-Open-Unmix:


The training curves can be seen in figure \ref{fig:networkloss}. These were taken from Tensorboard,\footnote{\url{https://www.tensorflow.org/tensorboard}} which is a visual web component of the Tensorflow deep learning framework (\cite{tensorflow, tensorflowsoft}) that is also compatible with PyTorch. It was used to visualize the loss curves of xumx-sliCQ during training.

\begin{figure}[ht]
	\centering
	\subfloat[Train loss]{\includegraphics[width=\textwidth]{./images-neural/train_loss.png}}
	\hspace{0.5em}
	\subfloat[Validation loss]{\includegraphics[width=\textwidth]{./images-neural/valid_loss.png}}
	\caption{Tensorboard loss curves for xumx-sliCQ, 1000 epochs}
	\label{fig:networkloss}
\end{figure}

\subsection{Demixing results}

The BSSv4 scores for the demixing results, computed on the MUSDB18-HQ (\cite{musdb18hq}) dataset's test split, are shown in figure \ref{fig:bssboxplot}. Open-Unmix is the reference model (\cite{umx}) and is labelled ``umx'' in the boxplot. CrossNet-Open-Unmix 

\begin{figure}[ht]
	\centering
	\includegraphics[height=5cm]{./images-bss/boxplot_full.png}
	\caption{Boxplot of full MUSDB18-HQ test set evaluation of Open-Unmix vs. CrossNet-Open-Unmix vs. xumx-sliCQ}
	\label{fig:bssboxplot}
\end{figure}

\subsection{Inference performance analysis}

\subsection{ISMIR 2021 Music Demixing Challenge}

\ichfeedback{i know this shouldn't be a core part of the thesis, but it's still fun to describe?}

\end{document}
