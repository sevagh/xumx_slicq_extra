\documentclass[report.tex]{subfiles}
\begin{document}

\section{Experiment}
\label{sec:experiment}

\subsection{Performance benchmarks for GPU accelerations}

\chaptertodo{
	cupy accel of sigsep-mus-eval\\
	pytorch accel of nsgt/slicqt
}

\newpagefill

\subsection{sliCQT parameters with MPI oracle}

\begin{figure}[ht]
	\centering
	\subfloat[Good sliCQT -- Bark, 262 bins, 32.9--22050 Hz]{\includegraphics[width=\textwidth]{./images-gspi/slicqt_good.png}}\\
	\subfloat[Bad sliCQT -- Constant-Q/log, 142 bins, 129.7--22050 Hz]{\includegraphics[width=\textwidth]{./images-gspi/slicqt_bad.png}}
	\caption{sliCQTs from the oracle MPI hyperparameter search}
	\label{fig:bipolarslicqs}
\end{figure}

\begin{figure}[ht]
	\centering
	\includegraphics[width=0.95\textwidth]{./images-bss/oracle_boxplot.pdf}
	\caption{Boxplot for oracle mask evaluations}
	\label{fig:oraclebssboxplot}
\end{figure}

\newpagefill

\subsection{Neural network architecture and training}

The training parameters of xumx-sliCQ were kept similar to Open-Unmix and CrossNet-Open-Unmix:

The training curves can be seen in figure \ref{fig:networkloss}. These were taken from Tensorboard,\footnote{\url{https://www.tensorflow.org/tensorboard}} which is a visual web component of the Tensorflow deep learning framework (\cite{tensorflow, tensorflowsoft}) that is also compatible with PyTorch. It was used to visualize the loss curves of xumx-sliCQ during training.

\begin{figure}[ht]
	\centering
	\subfloat[Train loss]{\includegraphics[width=\textwidth]{./images-neural/train_loss.png}}
	\hspace{0.5em}
	\subfloat[Validation loss]{\includegraphics[width=\textwidth]{./images-neural/valid_loss.png}}
	\caption{Tensorboard loss curves for xumx-sliCQ, 1000 epochs}
	\label{fig:networkloss}
\end{figure}

\newpagefill

\subsection{Demixing results}

The BSSv4 scores for the demixing results, computed on the MUSDB18-HQ (\cite{musdb18hq}) dataset's test split, are shown in figure \ref{fig:bssboxplot}. Table \ref{table:bsseval} contains the details of every model evaluated in the boxplot with their label.

\begin{table}[ht]
	\centering
	\begin{tabular}{ |l|l|p{4cm}|p{4cm}| }
	 \hline
	  Project & Boxplot label & Code repository & Pretrained model \\
	 \hline
	 \hline
		\makecell[l]{Open-Unmix \\ \textcite{umx}} & umx & \url{https://github.com/sigsep/open-unmix-pytorch} & \url{https://zenodo.org/record/3370489} (UMX-HQ) \\
	 \hline
		\makecell[l]{CrossNet-Open-Unmix \\ \textcite{xumx}} & xumx & \url{https://github.com/sony/ai-research-code/tree/master/x-umx} & \url{https://nnabla.org/pretrained-models/ai-research-code/x-umx/x-umx.h5} (X-UMX) \\
	 \hline
		\makecell[l]{xumx-sliCQ \\ (this paper)} & sliCQ & \url{https://github.com/sevagh/xumx-sliCQ} & \url{https://github.com/sevagh/xumx-sliCQ/tree/main/pretrained-model} \\
	 \hline
\end{tabular}
	\caption{Evaluated pretrained models in the BSS boxplot}
	\label{table:bsseval}
\end{table}

\begin{figure}[ht]
	\centering
	\includegraphics[width=0.9\textwidth]{./images-bss/boxplot_full.pdf}
	\caption{Boxplot of Open-Unmix vs. CrossNet-Open-Unmix vs. xumx-sliCQ}
	\label{fig:bssboxplot}
\end{figure}

\newpagefill

\subsubsection{Wiener-EM with the STFT and sliCQT}

\newpagefill

\subsection{Inference performance analysis}

\chaptertodo{
	use ISMIR 2021 Music Demixing Challenge timeout rules
}

\end{document}
