\documentclass[letter,12pt]{article}
\usepackage[svgnames]{xcolor}
\usepackage[left=2.5cm, right=2.5cm, top=2cm, bottom=2.5cm]{geometry}
\usepackage{hyperref}
\hypersetup{
  colorlinks   = true, %Colours links instead of ugly boxes
  urlcolor     = DarkBlue, %Colour for external hyperlinks
  linkcolor    = black, %Colour of internal links
  citecolor   = black %Colour of citations
}
\usepackage{setspace}
\renewcommand{\topfraction}{0.85}
\renewcommand{\textfraction}{0.1}
\renewcommand{\floatpagefraction}{0.75}
%\usepackage[backend=biber,authordate,annotation,url=false,doi=false]{biblatex-chicago}
%\addbibresource{citations.bib}
\usepackage[
    %backend=biber, 
    natbib=true,
    style=numeric,
    sorting=none,
]{biblatex}
\addbibresource{citations.bib}

\newcommand\ThesisTitle{Music Source Separation with the sliCQ Transform}


\newenvironment{tight_enumerate}{
\begin{enumerate}
  \setlength{\itemsep}{0pt}
  \setlength{\parskip}{0pt}
}{\end{enumerate}}

%\renewcommand*{\postvolpunct}{\addcolon\addspace}

\title{\vspace{-2.25em}\textbf{\ThesisTitle}\vspace{-0.75em}}
\author{Sevag Hanssian}
\date{
\vspace{-0.75em}
\small{
Distributed Digital Music Archives \& Libraries Lab\\
Schulich School of Music, McGill University, Montr{\'e}al, Canada\\
}
\footnotesize{sevag.hanssian@mail.mcgill.ca}
\vspace{-1.25em}
}

\begin{document}

\maketitle
\thispagestyle{empty}

Music demixing is the task of decomposing a song into its constituent sources, which are typically isolated instruments (e.g., drums, bass, and vocals). Open-Unmix (UMX) \parencite{umx} and CrossNet-Open-Unmix (X-UMX) \parencite{xumx} are models for music demixing that use the Short-Time Fourier Transform (STFT) to represent musical signals, but the time-frequency uncertainty principle states that the STFT of a signal cannot have maximal resolution in both time and frequency \parencite{gabor1946}. The tradeoff in time-frequency resolution can significantly affect music demixing results \parencite{tftradeoff1}. The STFT is computed by applying the Discrete Fourier Transform on fixed-size windows of the input signal, but for auditory and musical considerations, variable-sized windows are preferred to vary the time-frequency resolution by frequency region \parencite{doerflerphd}. Our proposed adaptation of UMX and X-UMX, called xumx-sliCQ,\footnote{\url{https://github.com/sevagh/xumx-sliCQ}} replaces the STFT with the sliCQT \parencite{slicq}, an invertible transform with varying time-frequency resolution. It uses a convolutional network architecture \parencite{plumbley2} trained on the MUSDB18-HQ \parencite{musdb18hq} dataset. On the test set, xumx-sliCQ achieved a median SDR of 3.6 dB versus the 4.64 dB of UMX and 5.54 dB of X-UMX, unfortunately performing worse than the original STFT-based models.

\begingroup
\setstretch{0.9}
\setlength\bibitemsep{0.015em}
%\printbibheading[title={References},heading=bibnumbered]
%\printbibliography[heading=none]
\printbibliography
\endgroup

\end{document}
