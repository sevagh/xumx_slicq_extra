\documentclass[usenames,dvipsnames]{beamer}
\usetheme{Boadilla}
\usepackage{hyperref}
\usepackage{graphicx}
\usepackage{multimedia}
\usepackage{fancyvrb}
\usepackage{multicol}
\usepackage{optparams}
\usepackage{adjustbox}
\usepackage{tikz}
\usetikzlibrary{shapes,positioning}
\newcommand{\foo}{\hspace{-2.3pt}$\bullet$ \hspace{5pt}}
\usepackage{subfig}
\usepackage[
    backend=biber,
    natbib=true,
    style=numeric,
    sorting=none,
    style=verbose-ibid,
    maxcitenames=1, %remove this outside of toy presentations
]{biblatex}
\addbibresource{citations.bib}
\usepackage{pgfpages}
\usepackage{xcolor}
\definecolor{ao(english)}{rgb}{0.0, 0.5, 0.0}
\definecolor{burgundy}{rgb}{0.5, 0.0, 0.13}
%\setbeameroption{show notes}
%\setbeameroption{show notes on second screen=right}
\setbeameroption{hide notes}
\newcommand\ThesisTitle{Music Source Separation with the sliCQ Transform}


\title{Improving Open-Unmix for Music Source Separation}
\subtitle{Brainstorming session}
\author{Sevag Hanssian}
\institute{DDMAL, McGill}
\setbeamertemplate{navigation symbols}{}

\AtEveryBibitem{%
  \clearfield{pages}%
  \clearfield{volume}%
  \clearfield{number}%
  \clearlist{journal}%
  \clearfield{booktitle}%
}

\renewbibmacro{in:}{}

\AtEveryCitekey{%
  \clearfield{pages}%
  \clearfield{volume}%
  \clearfield{number}%
  \clearfield{doi}%
  \clearfield{journal}%
  \clearlist{journal}%
  \clearfield{booktitle}%
  \clearfield{isbn}%
  \clearfield{title}%
  \clearfield{url}%
\ifentrytype{article}{
    \clearfield{journal}%
}{}
\ifentrytype{inproceedings}{
    \clearfield{booktitle}%
}{}
}

\begin{document}

\begin{frame}
\maketitle
\end{frame}

\begin{frame}[fragile]
	\frametitle{Music Source Separation}
	Task introduction \footcite{musicsepgood}:
	\begin{enumerate}
		\item
			When producing music, instruments are recorded separately (aka stems), and are combined (mixing + mastering) to make mixed song -- commonly 4 sources, bass, drums, vocals, other, deriving from MUSDB18-HQ dataset \footcite{musdb18hq}. E.g.:
			\begin{verbatim}
			# ls MUSDB18-HQ/train/Night\ Panther\ -\ Fire/
			vocals.wav mixture.wav drums.wav other.wav bass.wav
			\end{verbatim}
		\item
			On the flip side, given a mixed song, we want to decompose it back into its stems (without having access to the stems)
		\item
			How? Exploit time-frequency characteristics of the sources that distinguish them from each other. Simplistic approach \footcite{fitzgerald1}: harmonic sources have long sustained frequencies, percussive sources are transient and broadband, i.e. tall, short-lived spikes in time
	\end{enumerate}
\end{frame}

\begin{frame}
	\frametitle{Visualizations}
	\begin{figure}[ht]
		\centering
		\subfloat{\includegraphics[height=2.5cm]{./mixing.png}}
		\hspace{0.1em}
		\subfloat{\includegraphics[height=2.5cm]{./demixing.png}}
		\caption{Mixing and ``demixing'' or source separation}
	\end{figure}
	\begin{figure}[ht]
		\centering
		\subfloat{\includegraphics[height=2cm]{./mixedspecgram.png}}
		\hspace{0.1em}
		\subfloat{\includegraphics[height=2cm]{./harm_soft.png}}
		\hspace{0.1em}
		\subfloat{\includegraphics[height=2cm]{./perc_soft.png}}
		\caption{HPSS example}
	\end{figure}
\end{frame}

\begin{frame}[fragile]
	\frametitle{Neural networks for Music Source Separation}
	Task introduction \footcite{musicsepgood}:
	\begin{enumerate}
		\item
			\textbf{Ground truth:} source stems that are combined to make mixed song -- commonly 4 sources, bass, drums, vocals, other. Also very commonly 1 network per source
		\item
			\textbf{Network inputs:} mixed waveform and 1 source -- direct waveform, or a TFR (time-frequency representation)  e.g. STFT
		\item
			\textbf{Estimate source:} given the mixed input, learn how to extract a source. Common approach: learnable magnitude-STFT mask, e.g. same shape as spectrogram but $\in [0.0, 1.0]$, to multiply mixed spectrogram with
		\item
			\textbf{Back to waveform:} after estimating the source, convert it back to waveform -- can use masking or mix-phase inversion. Mixture supplies the STFT phase, estimates only use magnitude
	\end{enumerate}
\end{frame}

\begin{frame}
	\frametitle{Visualizations}
	\begin{figure}[ht]
		\includegraphics[height=2.5cm]{./mask_simple.png}
		\caption{Example of a binary mask applied to an STFT}
	\end{figure}
	\begin{figure}[ht]
		\includegraphics[height=2.5cm]{./whynophase.png}
		\caption{Why we don't try to learn phase in neural networks}
	\end{figure}
\end{frame}

\begin{frame}
	\frametitle{Open-Unmix}
	Very open source, near-SOTA music source separation. Intended to be a ``platform'' for future MSS research.
	\begin{figure}[ht]
		\includegraphics[height=3cm]{./umx1.png}
		\caption{Open-Unmix diagram}
	\end{figure}
\end{frame}

\begin{frame}
	\frametitle{Open-Unmix network}
	UMX simplied (example for vocals, but identical for other stems):
	\begin{enumerate}
		\item
			\textbf{x, y training example:} x = mix waveform, y = vocal stem (6 second sequence)
		\item
			\textbf{Network:} magnitude STFT = $abs(STFT(x))$ = $\text{Time} \times \text{Frequency}$\\E.g. for STFT with 4096-sized window, on 6 seconds of audio = $258 \text{ time frames} \times 2049 \text{ frequencies}$
		\item
			\textbf{Linear encoder:} 2049 frequencies $\rightarrow$ 512 (compressed frequencies)
			(+ batchnorm + tanh)
		\item
			\textbf{Recurrence}: BLSTM, hidden = 512, 3 layers, dropout + skip-conn\\
			258 frames of 512 compressed frequencies pass through the LSTM
		\item
			\textbf{Linear decoder:}
			\begin{enumerate}
				\item
					1024 (LSTM skip conn) $\rightarrow$ 512 (compressed frequencies)\\(+ batchnorm + relu)
				\item
					512 $\rightarrow$ 2049 (original frequencies)\\(+ batchnorm + relu)
			\end{enumerate}
		\item
			\textbf{Apply positive mask on mix:} return $\text{mix} \times \text{network output}$
		\item
			\textbf{Loss:} MSE of $abs(STFT(y))$ vs. $unmix(abs(STFT(x)))$ i.e. how identical is the estimate vocals magnitude STFT to the ground truth vocals magnitude STFT
	\end{enumerate}
\end{frame}

\begin{frame}
	\frametitle{Getting the stem waveform}
	Recall: output of unmix is the estimated \textbf{magnitude STFT} of the vocal stem\\\ \\

	Open-unmix uses phase of mixture:
	\begin{enumerate}
		\item
			\textbf{x, inference:} x = mix waveform
		\item
			\textbf{Estimate Ymag, vocals:}\\
			\qquad $X_{\text{mag}} = abs(\text{STFT}(x))$, $\hat{Y}_{\text{mag}} = \text{network}(X_{\text{mag}})$
		\item
			\textbf{Mix phase inversion:}\\
			\qquad $X_{\text{phase}} = atan2(\text{STFT}(x))$\\
			\qquad $\hat{Y}_{\text{complex}} = \hat{Y}_{\text{mag}} + j X_{\text{phase}} \rightarrow \hat{y} = \text{iSTFT}(\hat{Y}_{\text{complex}})$
		\item
			Evaluation: BSS (SDR, SAR, SIR, ISR) on $\hat{y}, y_{\text{groundtruth}}$
	\end{enumerate}
\end{frame}

\begin{frame}
	\frametitle{Oracle, or ideal, performance}
	Mix phase oracle (hasn't been evaluated before). Given ground truths, \textbf{how good can source separation be?} Assuming a perfect estimate from the neural network\\

	Using ground-truth instead of network predictions:
	\begin{enumerate}
		\item
			\textbf{x, y ground truth:} x = mix waveform, y = vocals
		\item
			\textbf{Ideal mix phase inversion:}\\
			\qquad $X_{\text{phase}} = atan2(\text{STFT}(x))$\\
			\qquad $Y_{\text{mag}} = abs(\text{STFT}(y))$\\
			\qquad $Y_{\text{ideal complex}} = Y_{\text{mag}} + j X_{\text{phase}} \rightarrow y_{\text{ideal}} = \text{iSTFT}(Y_{\text{ideal complex}})$
		\item
			Evaluation: BSS (SDR, SAR, SIR, ISR) on $y_{\text{ideal}}, y_{\text{groundtruth}}$
	\end{enumerate}
\end{frame}

\begin{frame}
	\frametitle{STFT time-frequency tradeoff -- solved by NSGT}
	\begin{figure}[ht]
		\centering
		\subfloat[STFT, 93ms window]{\includegraphics[height=2.5cm]{./tf_tradeoff_balasz2.png}}
		\hspace{0.5em}
		\subfloat[STFT, 6ms window]{\includegraphics[height=2.5cm]{./tf_tradeoff_balasz1.png}}\\
		\subfloat[NSGT, 6-93ms window]{\includegraphics[height=2.5cm]{./tf_tradeoff_balasz3.png}}
		\caption{STFT vs. NSGT}
		\label{fig:stfttradeoff}
	\end{figure}
\end{frame}

\begin{frame}
	\frametitle{Idea: choose NSGT configuration for ideal mix phase oracle}
	Original idea which kicked off entire project. STFT is minimally configurable: window and hop size only.

	NSGT:
	\begin{enumerate}
		\item
			Choice of frequency scale: Constant-Q/logarithmic, Variable-Q, Mel, Bark
		\item
			Choice of fmin (10.0-130.0 Hz in \footcite{balazs})
		\item
			Choice of fmax (nyquist for performance, typically)
		\item
			Choice of total frequency bins (or ``bins per octave'' in cqt language)
	\end{enumerate}
	Slice-wise (e.g. windowed, familiar to the STFT) variant = ``sliCQ'' NSGT\\

	NSGT is not too special or exotic -- it's an FFT applied with varying window sizes (the special trait of the NSGT is the scheme to compute the perfect inverse using frame theory)
\end{frame}

\begin{frame}
	\frametitle{After some hyperparam searches}
	Using SDR only (the most important score):\\
	STFT, UMX default (window = 4096): vocals 8.62, drums 7.07, bass 6.68, other 6.87\\

	NSGT (with maximum slice length capped at 8192 frames):\\

	Bass NSGT = Bark, 105 bins, 25.4-22050 Hz, slice = 7180 = 8.47\\
	Drums NSGT = Mel, 104 bins, 49.3-22050 Hz, slice = 7108 = 10.5\\
	Other NSGT = Bark, 64 bins, 90.0-22050 Hz, slice = 4416 = 13.94\\
	Vocals NSGT = Mel, 116 bins, 37.7-22050 Hz, slice = 8024 = 9.83\\\ \\


	\textbf{Hypothesis}:
	\begin{enumerate}
		\item
			If one can prepare a neural network that can estimate the magnitude-sliCQ-NSGT
		\item
			Then using the above configurations can get significant SDR gains since they perform considerably better in the mix-phase oracle
		\item
			UMX-STFT: estimate abs(STFT-4096) $\rightarrow$ \\
			\qquad UMX-NSGT: estimate abs(NSGT-Bark-105-25.4Hz-7180)
	\end{enumerate}
\end{frame}

\begin{frame}[fragile]
	\frametitle{NSGT-sliCQ detail}

	\begin{enumerate}
		\item
			x, mixed waveform
		\item
			nsgt = NSGT(fscale=mel, fbins=105, fmin=25.4, fmax=22050, sllen=7180)
		\item
			X = nsgt.forward(x), complex 3d slicq
		\item
			Shape: $\text{Slice} \times \text{Frequency bin} \times \text{Time-in-slice}$
			e.g. $67 \times 117 \times 234$
		\item
			2d shape:
			\begin{enumerate}
				\item
					Flatten: $(67 \times 234) \times 117$ time-freq = $15678 \times 117$
				\item
					50\% overlap: $7839 \times 117$ -- noninvertible
			\end{enumerate}
		\item
			117 frequency bins are nonlinear on the desired scale:
			\begin{verbatim}
			 25.4           49.84    75.1   101.2 ...
			\end{verbatim}
	\end{enumerate}
\end{frame}

\begin{frame}
	\frametitle{NSGT-sliCQ inside Open-Unmix}
	NSGT-sliCQ choices:
	\begin{itemize}
		\item
			Reduce to 2d (overlap-add or flatten) -- same dimensionality as an STFT
		\item
			Treat as 3-dimensional sequence of 2d time-frequency frames!\\
			2-d spectrogram = image (RGB x width x height $\rightarrow$ mono/stereo x time x frequency)\\
			3-d sliCQ = video (temporal frame x mono/stereo x (2d time x frequency)\\
			$\text{Slice} \times (\text{Frequency} \times \text{Time-in-slice})$
	\end{itemize}
\end{frame}

\begin{frame}
	\frametitle{Open-Unmix with 2d sliCQ + LSTM}
	UMX-sliCQ 2d:
	\begin{enumerate}
		\item
			\textbf{x, y training example:} x = mix waveform, y = vocal stem (6 second sequence)
		\item
			\textbf{Network:} magnitude NSGT = $abs(NSGT(x))$ = $\text{Slices} \times \text{Frequency} \times \text{Time-in-slice}$\\E.g. for NSGT-mel-115-25.4hz, on 6 seconds of audio = $67 \text{ slices} \times 117 \text{ frequencies} \times 234 \text{ time-in-slice bins} $
		\item
			\textbf{Flatten:} $67 \text{ slices} \times 117 \text{ frequencies} \times 234 \text{ time-in-slice bins} \rightarrow 15678 \text{ time} \times 117 \text { frequencies}$\\
			\textbf{\textcolor{red}{LSTMs can deal with overlapping sequences, right?}}
		\item
			\textbf{Recurrence}: BLSTM, hidden = 117, 3 layers, dropout + skip-conn\\
			15678 frames of 117 frequencies pass through the LSTM\\
			\textbf{contrast with} 258 frames of 512 compressed frequencies of STFT
			\textbf{\textcolor{red}{huge sequence length is not good -- slow training, bad gradients, never worked for me}}
	\end{enumerate}
\end{frame}

\begin{frame}
	\frametitle{Open-Unmix with 3d sliCQ + LSTM}
	UMX-sliCQ 3d:
	\begin{enumerate}
		\item
			\textbf{x, y training example:} x = mix waveform, y = vocal stem (6 second sequence)
		\item
			\textbf{Network:} magnitude NSGT = $abs(NSGT(x))$ = $\text{Slices} \times \text{Frequency} \times \text{Time-in-slice}$\\E.g. for NSGT-mel-115-25.4hz, on 6 seconds of audio = $67 \text{ slices} \times 117 \text{ frequencies} \times 234 \text{ time-in-slice bins} $
		\item
			\textbf{Flatten differently:} $67 \text{ slices} \times 117 \text{ frequencies} \times 234 \text{ time-in-slice bins} \rightarrow 67 \text{ slices} \times (117 \text { frequencies} \times 234 \text { time-in-slice} = 27378)$\\
			\textbf{LSTM is now taking the temporally evolving sequence of ``1D'' flattened sliCQ tf frames)}
		\item
			\textbf{Recurrence}: BLSTM, hidden = 27378, 3 layers, dropout + skip-conn\\
			67 frames of 27378 sliCQ coefficients pass through the LSTM\\
			\textbf{contrast with} 258 frames of 512 compressed frequencies of STFT
			\textbf{\textcolor{red}{reducing the 1D dimensionality to not have a hidden size of 27378 -- linear encoder?}}
	\end{enumerate}
\end{frame}

\begin{frame}
	\frametitle{Open-Unmix with 3d sliCQ + Conv-LSTM}
	UMX-sliCQ 3d:
	\begin{enumerate}
		\item
			\textbf{x, y training example:} x = mix waveform, y = vocal stem
		\item
			\textbf{Network:} magnitude NSGT = $abs(NSGT(x))$ = $\text{Slices} \times \text{Frequency} \times \text{Time-in-slice}$\\E.g. for NSGT-mel-115-25.4hz, on 6 seconds of audio = $67 \text{ slices} \times 117 \text{ frequencies} \times 234 \text{ time-in-slice bins} $
		\item
			\textbf{Conv3d encoder:} Use $1 \times T \times F$ kernels to not touch the slice/temporal dimension:\\
			$67 \text{ slices} \times 117 \text{ frequencies} \times 234 \text{ time-in-slice bins} \rightarrow$ strided convolution layers $\rightarrow 67 \text{ slices} \times (6 \text { fconv} \times 10 \text { tconv} = 60)$\\
			\textbf{LSTM is now taking the temporally evolving sequence of ``1D'' flattened convencoder(sliCQ)}
		\item
			\textbf{Recurrence}: BLSTM, hidden = 60, 3 layers, dropout + skip-conn\\
			67 frames of 60 conv3d(sliCQ) coefficients pass through the LSTM
		\item
			\textbf{ConvTranspose3d decoder:} grow $60 = 6 \times 10$ back to $117 \times 234$

			\textbf{``feels right'' but doesn't generalize - \textcolor{red}{convolution parameter selection - kernels, layers, hidden channels}}
	\end{enumerate}
\end{frame}

\begin{frame}
	\frametitle{Open-Unmix with 3d sliCQ + Convolutions, no LSTM}
	UMX-sliCQ 3d:
	\begin{enumerate}
		\item
			\textbf{x, y training example:} x = mix waveform, y = vocal stem (6 second sequence)
		\item
			\textbf{Network:} magnitude NSGT = $abs(NSGT(x))$ = $\text{Slices} \times \text{Frequency} \times \text{Time-in-slice}$\\E.g. for NSGT-mel-115-25.4hz, on 6 seconds of audio = $67 \text{ slices} \times 117 \text{ frequencies} \times 234 \text{ time-in-slice bins} $
		\item
			\textbf{Conv3d encoder:} Use $S \times T \times F$ kernels to also find patterns across the slice dimension
			$67 \text{ slices} \times 117 \text{ frequencies} \times 234 \text{ time-in-slice bins} \rightarrow$ strided convolution layers $\rightarrow 4 \text{ slices} \times (6 \text { fconv} \times 10 \text { tconv} = 60)$\\
		\item
			\textbf{ConvTranspose3d decoder:} grow $4 \times 6 \times 10$ back to $67 \times 117 \times 234$
	\end{enumerate}
	\textbf{\textcolor{red}{based more on Convolutional autoencoder/decoder theory, not an adaptation of UMX anymore}}\\
	\textbf{\textcolor{red}{no luck}}\\
\end{frame}

\end{document}
