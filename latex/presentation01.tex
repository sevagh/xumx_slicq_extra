\documentclass[usenames,dvipsnames]{beamer}
\usetheme{Boadilla}
\usepackage{hyperref}
\usepackage{graphicx}
\usepackage{multimedia}
\usepackage{fancyvrb}
\usepackage{multicol}
\usepackage{optparams}
\usepackage{adjustbox}
\usepackage{tikz}
\usetikzlibrary{shapes,positioning}
\newcommand{\foo}{\hspace{-2.3pt}$\bullet$ \hspace{5pt}}
\usepackage{subfig}
\usepackage[
    backend=biber,
    natbib=true,
    style=numeric,
    sorting=none,
    style=verbose-ibid,
    maxcitenames=1, %remove this outside of toy presentations
]{biblatex}
\addbibresource{citations.bib}
\usepackage{pgfpages}
\usepackage{xcolor}
\definecolor{ao(english)}{rgb}{0.0, 0.5, 0.0}
\definecolor{burgundy}{rgb}{0.5, 0.0, 0.13}
%\setbeameroption{show notes}
%\setbeameroption{show notes on second screen=right}
\setbeameroption{hide notes}
\newcommand\ThesisTitle{Music Source Separation with the sliCQ Transform}


\title{\ThesisTitle}
\subtitle{MA Thesis research proposal}
\author{Sevag Hanssian}
\institute{DDMAL, McGill}
\setbeamertemplate{navigation symbols}{}

\AtEveryBibitem{%
  \clearfield{pages}%
  \clearfield{volume}%
  \clearfield{number}%
  \clearlist{journal}%
  \clearfield{booktitle}%
}

\renewbibmacro{in:}{}

\AtEveryCitekey{%
  \clearfield{pages}%
  \clearfield{volume}%
  \clearfield{number}%
  \clearfield{doi}%
  \clearfield{journal}%
  \clearlist{journal}%
  \clearfield{booktitle}%
  \clearfield{isbn}%
  \clearfield{title}%
  \clearfield{url}%
\ifentrytype{article}{
    \clearfield{journal}%
}{}
\ifentrytype{inproceedings}{
    \clearfield{booktitle}%
}{}
}

\begin{document}

\begin{frame}
\maketitle
\end{frame}

\begin{frame}
	\frametitle{My own goals}
	Motivating musical applications based on what I like working on: Music source separation -- why?
	\begin{enumerate}
		\item
			DSP/ASP-oriented (my personal favorite)
		\item
			Ground truths are not opinion-based:\\
			$x_{a} + x_{b} = x_{\text{mix}}, \text{MusicSourceSeparation}(x_{\text{mix}}) = [\hat{x_{a}}, \hat{x_{b}}]$\\
			(however, evaluation measures of course have subjective/opinion-based counterparts)
		\item
			I like beat tracking/onset detection, but the subjectivity of ground truths, inaccessible ISMIR/MIREX datasets are off-putting
	\end{enumerate}
	What I want to do:
	\begin{enumerate}
		\item
			Hypothesis -- ``Replace the STFT with NSGT in spectrogram-based music source separation, and get better results''
		\item
			Why NSGT? time-frequency transform with perfect inverse, which can be used with arbitrary frequency scales, e.g. mel/bark/octave/log (octave/log NSGT == CQT or CQ-NSGT)
		\item
			Get a publishable result by contributing to music source separation
	\end{enumerate}
\end{frame}

\begin{frame}
	\frametitle{Problems with the STFT}
	\begin{itemize}
		\item
			Linear frequency scale -- not natural, either musically or psychoacoustically
		\item
			Fixed time-frequency resolution -- must pick small window for good temporal resolution, or large window for good harmonic resolution - not both
		\item
			Consequence of the time-frequency uncertainty principle
		\item
			In practice, this means picking the window size of STFT has an impact on the problem being studied, and that we must either commit to a single, not-one-size-fits-all STFT size (and sacrifice some potential gains), or use multiple sizes of STFT in the same model (this is common, e.g., Sebastian B{\"o}ck's beat and onset algorithms)
	\end{itemize}
\end{frame}

\begin{frame}
	\frametitle{NSGT}
	According to Monika Do{\"e}rfler, musical signals have characteristics that lead to conflicting requirements in the STFT \footcite{doerflerphd}:
	\begin{itemize}
		\item
			In the low frequency region, music needs to be analyzed with long-duration windows for a high frequency resolution, as bass notes lay the harmonic basis of a song
		\item
			The high frequency region contains transients and broadband sounds, which are useful for timbre and rhythm. These have fast attacks and decays and need to be analyzed with short-duration windows
	\end{itemize}
	Need at least 2 STFTs (short and long window) due to the fixed time-frequency resolution of each STFT -- \textbf{or, one NSGT}\\
	The Nonstationary Gabor transform (NSGT) \footcite{balazs} is a time-frequency transform with a varying time-frequency resolution and a perfect inverse.\\
	Would also be nice to get some real musical support for these statements (not just perpetuating a single paper by a DSP expert, but corroborating it in musical literature) -- maybe i can ask the lab (after my own efforts)?
\end{frame}

\begin{frame}
	\frametitle{Problems with the STFT -- visual}
	\begin{figure}[ht]
		\centering
		\subfloat[Wide window STFT (93ms)]{\includegraphics[height=3.5cm]{./tf_tradeoff_balasz2.png}}
		\hspace{0.5em}
		\subfloat[Narrow window STFT (6ms)]{\includegraphics[height=3.5cm]{./tf_tradeoff_balasz1.png}}
		\caption{Different window size STFT spectrograms of a glockenspiel signal}
		\label{fig:stfttradeoff}
	\end{figure}
	Note blurry frequency or blurry time -- window size STFT tradeoff
\end{frame}

\begin{frame}
	\frametitle{NSGT good for music -- visual}
	\begin{figure}[ht]
		\centering
		\includegraphics[height=5cm]{./tf_tradeoff_balasz3.png}
		\caption{NSGT spectrogram of a glockenspiel signal, varying window (6-93ms)}
		\label{fig:nsgttradeoff}
	\end{figure}
\end{frame}

\begin{frame}
	\frametitle{Judith Brown CQT -- classic paper}
	NSGT is motivated by the CQT, classic paper by \footcite{jbrown}, creating a transform which maintained a constant ratio of frequency to frequency resolution, because harmonics of f0 created by musical instruments have a consistent spacing in the log scale (\textit{constant pattern})
	\begin{figure}[ht]
		\vspace{-0.75em}
		\centering
		\subfloat[Linear-frequency DFT]{\includegraphics[height=3.5cm]{../latex/violindft.png}}
		\hspace{0.5em}
		\subfloat[Constant-Q transform]{\includegraphics[height=3.5cm]{../latex/violincqt.png}}
		\caption{Violin playing the diatonic scale, $G_{3} \text{(196Hz)} - G_{5} \text{(784Hz)}$}
	\end{figure}
\end{frame}

\begin{frame}
	\frametitle{Klapuri CQT}
	Klapuri's CQT implementation is used in librosa, and up until the NSGT, in MATLAB. According to \footcite{klapuricqt}, ``the CQT is well-motivated from both musical and perceptual viewpoints'':
	\begin{enumerate}
		\item
			\textbf{Nonlinear frequency spacing:} Fundamental frequencies (F0s) of the tones in Western music are geometrically spaced. From auditory perspective, the frequency resolution of the peripheral hearing system of humans is approx constant-Q (cites B. C. J. Moore)
		 \item
			 \textbf{Sharp transient/temporal resolution:} From perceptual audio coding (cites AAC codec from ISO/IEC), shortest transform window lengths are $\approx$ 3ms to retain high quality, whereas higher frequency resolution is required to carry out coding at low frequencies \footcite{cqtransient}
		\item
			Contrast with the conventional DFT which has linearly spaced frequency bins and cannot satisfy the \textbf{varying tf resolution} requirements over the wide range of audible frequencies
	\end{enumerate}
\end{frame}

\begin{frame}
	\frametitle{Why study NSGT over CQT?}
	One of the most important applications of the NSGT is an invertible CQT (or CQ-NSGT) \footcite{invertiblecqt} \footcite{variableq1}\\
	However, we are not limited to the constant-Q scale, can use arbitrary scales, including Bark, Mel, Variable-Q (by ERB)\\
	So, NSGT lets us branch out of the manual selection of the logarithmic/constant-Q scale - what if other scales are useful too?\\
	Also we might lose the stigma of the ``non-invertible CQT boogeyman'' (but I don't think I can cite or prove this -- however I did find a paper which noted the pre-NSGT CQT's lack of inverse \footcite{lackinverse}
\end{frame}

\begin{frame}
	\frametitle{STFT spectrogram in music source separation}
	\begin{itemize}
		\item
			Many music source separation algorithms (kernel additive module, nonnegative matrix factorization, and recent machine learning models) use the STFT spectrogram as input
		\item
			Sparsity of the sources is important -- e.g., drums are vertical lines in the spectrogram, guitar is horizontal lines, etc. \footcite{musicsepgood} -- indicating sparser transforms might be a good idea (i.e. an NSGT which is less blurry has more sparsity)
	\end{itemize}
	\begin{figure}[ht]
		\vspace{-1em}
		\centering
		\subfloat{\includegraphics[height=3.0cm]{./mss1.png}}
		\subfloat{\includegraphics[height=3.1cm]{./mss2.png}}
		\caption{MSS spectrogram sparsity}
	\label{fig:sepgood}
\end{figure}

\end{frame}

\begin{frame}
	\frametitle{Why NSGT in music source separation?}
	\begin{itemize}
		\item
			Some algorithms for MSS use 2 window sizes (small and large) for percussive and harmonic separation \footcite{fitzgerald2, driedger} -- evidence of TF tradeoff
		\item
			Paper on how the window size of STFT affects results of speech and music separation \footcite{musicsepwindow}, using the CQT in speech \footcite{cqtseparation} and music source separation \footcite{fitzgerald2} with positive results, on using a warped/nonlinear frequency resolution STFT successfully in music source separation \footcite{bettermusicsep}
		\item
			 Open-Unmix is a near-SOTA benchmark model for music source separation based on the STFT spectrogram -- good candidate to study \footcite{umx} It is trained only on MUSDB18-HQ (open dataset -- good) and it separates into 4 sources, based on the stems of MUSDB18-HQ -- drums, bass, vocals, other. Designed for open, reproducible research
	\end{itemize}
\end{frame}

\begin{frame}
	\frametitle{Why NSGT in music source separation?}
	\begin{itemize}
		\item
			NSGT (mix of small + large window, arbitrary nonlinear frequency scale which can be used for CQT or any other) is worth studying
		\item
			If it shows promise in Open-Unmix, maybe other STFT-based models (hint: that's 99\% of them) can use NSGT for interesting results. i.e. we  may be able to ``retrofit'' STFT-based models with NSGT, and compete with newer waveform-based neural networks \footcite{demucs}
		 \item
			 \textbf{Hypothesis:} music source separation depends on sparsity of musical sources \footcite{musicsepgood}. NSGT allows a sharper time-frequency resolution in the frequency regions of interest, representing musical signals better. worth exploring
	\end{itemize}
\end{frame}

\begin{frame}
	\frametitle{Thesis sections}
	\begin{enumerate}
		\item
			Chapter 1 -- intro, motivation, previous work, objectives, etc.
		\item
			Chapter 2 -- waveforms, digital signals, time-frequency analysis, STFT, time-frequency uncertainty principle, demonstration on musical signals
		\item
			Chapter 3 -- NSGT, CQT, frequency scales of interest (e.g. constant-Q == pitch scale), demonstration on musical signals
		\item
			Chapter 4 -- Music source separation, goals, task, survey, evaluation measures (BSS, PEASS, MOS), datasets, spectral masking/STFT-based methods, contemporary waveform neural networks
		\item
			Chapter 5 -- NSGT in music source separation
		\item
			Chapter 6 -- conclusion, discussion, outlook, etc.
	\end{enumerate}
\end{frame}

\begin{frame}
	\frametitle{NSGT for music source separation}
	Idea: replace STFT with NSGT in source separation (\textcolor{ForestGreen}{\textbf{VALIDATED}})\\
	We can first focus on the IRM -- (oracle) ideal ratio mask, representing the theoretical  best performance of spectral masking based music source separation algorithms \footcite{irm, sisec2018, vincent07}\\
	\begin{enumerate}
		\item
			Creation of a new oracle mask for ``Mixed Phase Inversion''\href{https://github.com/sigsep/open-unmix-pytorch/issues/83}{https://github.com/sigsep/open-unmix-pytorch/issues/83} -- use oracles of mixed phase inversion + soft mask oracle (both strategies used by UMX)
		\item
			Beat the STFT oracle performance with NSGT (shows theoretical max performance, but not an actual working separation system yet) -- search for the best config
		\item
			Adapt Open-Unmix (recent modern SOTA spectrogram-based separation network) to use NSGT (aka UMX-NSGT) and beat Demucs (waveform-based model which beats Open-Unmix) -- check both MPI (mixed phase inversion) and IRM1 (soft mask)
	\end{enumerate}
\end{frame}

\begin{frame}
	\frametitle{Music source separation objective evaluation}
	Evaluations not based on human subjective analysis:
	\begin{itemize}
		\item
			Industry standard is BSS \footcite{bss} used in SiSec 2018 campaign for music source separation \footcite{sisec2018}. BSS is used for recently published top source separation models -- even in bleeding-edge waveform-based end-to-end neural waveform models: \href{https://github.com/facebookresearch/demucs}{https://github.com/facebookresearch/demucs}
		\item
			Industry standard dataset is MUSDB18 and MUSDB18-HQ \footcite{musdb18, musdb18hq}
		\item
			Used both in 622 so I'm equipped to evaluate
		\item
			Modern SOTA papers do some subjective evaluation -- but is it OK if I don't, to avoid the ethics and organization cost? can that be a known blind spot of my thesis?
	\end{itemize}
\end{frame}

\begin{frame}
	\frametitle{Music source separation -- work done so far}
	\begin{itemize}
		\item
			BSS is expensive computation. Optimized BSS and NSGT to use CuPy and reduce computation time for one track from 3:20m to 1m
		\item
			Check oracle performance of controls -- different sizes of STFT (1024, 2048, 4096, 8192, 16384) - bss scores per source
		\item
			Use Bayesian Optimization to search for the best parameter of NSGT in oracle performance. Useful for finding good hyperparameters for an expensive objective function. \textbf{Scales}: mel, Bark, Constant-Q (log) -- Variable-Q is a work in progress (needs to be implemented). \textbf{Bins} between 12-384 (CQT/NSGT papers stop at 96 bins - no good reason why \href{https://github.com/grrrr/nsgt/issues/26}{https://github.com/grrrr/nsgt/issues/26}). \textbf{fmin} between 15Hz (5 below 20Hz pysychoacoustic limit) and 60Hz (to include 57Hz, which is from klapuri's CQT paper). \textbf{fmax} - set it fixed to Nyquist for best performance and to fix one parameter (otherwise parameter space would be too large) - should I play with it further? klapuri uses 14.7 kHz. test this range?
		\item
			After oracles show that NSGT is promising/can beat STFT, adapt Open-Unmix to use NSGT -- train classic + NSGT and compare
	\end{itemize}
\end{frame}

\begin{frame}
	\frametitle{Publishing ideas}
	\begin{itemize}
		\item
			New oracle mask measurement of mixed phase inversion seems important. This is a common strategy used by recent source separation algorithms (also seen in CDAE by Plumbley). Unrelated to NSGT, but good contribution to music source separation. Nobody has measured or named this oracle yet
		\item
			Show NSGT surpassing oracles (both soft mask and mixed phase inversion) -- can kick off others replacing STFT with NSGT
		\item
			Adapt Open-Unmix to use NSGT (aka UMX-NSGT) and win back the performance crown
	\end{itemize}
\end{frame}

\begin{frame}
	\frametitle{Questions}
	\begin{itemize}
		\item
			Is this relevant to DDMAL? The focus on music will be to explain/justify the NSGT with appropriate musical references. Show how the constant-Q scale matches the Western pitch scale, etc.
		\item
			Also not going full Philippe -- no novel math or frame theory (besides enough to cover the background of the NSGT)
		\item
			One of my major questions is how musically important is music source separation? Should I explore whether its useful (beyond just the results in itself) -- e.g. as preprocessing for other MIR tasks?
		\item
			GPS grad progress report: For a first report, students complete the \textbf{Objectives} box only.
			Objectives:
			\begin{enumerate}
				\item
					Start running experiments and gathering results for music source separation with NSGT (thesis idea)
			\end{enumerate}
		\item
			Objectives for next meeting:\\
			\begin{enumerate}
				\item
					Completed oracle mask (mixed phase, and IRM1) evaluation with control STFTs
				\item
					Searched optimal NSGT to maximize each oracle (separate optimizations, along with the relevant analsys e.g. ``interestingly, NSGT with xyz was better at the mixed-phase oracle than NSGT with abc''
			\end{enumerate}
	\end{itemize}
\end{frame}

\end{document}
