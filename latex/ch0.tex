\documentclass[report.tex]{subfiles}
\begin{document}

\renewcommand{\abstractname}{Acknowledgement}
\begin{abstract}
\phantomsection
\addcontentsline{toc}{section}{\abstractname}
I would like to thank my supervisor, Professor Ichiro Fijunaga, for providing guidance throughout my master's degree and thesis. His rigor and attention to detail have made me a better researcher. I would like to thank my labmates who helped me work through many difficult parts; N{\'e}stor N{\'a}poles L{\'o}pez (who created Figure \ref{fig:glockwaveform}b), Timothy de Reuse, Emily Hopkins, and all the others. In the Music Technology department, Professor Philippe Depalle helped me hone my signal processing knowledge, and Professor Gary Scavone's emphasis on creativity allowed me to do my best work. I would like to thank my family and close friends for their unconditional support and encouragement.\\\ \\

Finally, I appreciate the SigSep organization\footnote{\url{https://github.com/sigsep}} on GitHub, which contains key open-source code projects for music source separation research. Their emphasis on providing high-quality code to accompany academic papers have fostered an environment that's welcoming to beginners like myself, and I'm happy that I was able to work on this thesis during their Music Demixing 2021 challenge \parencite{mdx21}.\footnote{\url{https://mdx-workshop.github.io/}}
\end{abstract}

\newpagefill

\renewcommand{\abstractname}{Abstract}
\begin{abstract}
\phantomsection
\addcontentsline{toc}{section}{\abstractname}
	Music source separation is the task of extracting an estimate of one or more isolated sources or instruments (for example, drums or vocals) from musical audio. The task of music demixing or unmixing considers the case where the musical audio is separated into an estimate of all of its constituent sources that can be summed back to the original mixture. Models for music demixing that use the Short-Time Fourier Transform (STFT) as their representation of music signals are popular and have achieved success in recent years. However, the fixed time-frequency resolution of the STFT, arising from the time-frequency uncertainty principle, requires a tradeoff in time-frequency resolution that can significantly affect music demixing results. The sliced Constant-Q Transform (sliCQT) is a time-frequency transform with varying time-frequency resolution that avoids the time-frequency tradeoff of the STFT. The model proposed by this thesis replaces the STFT with the sliCQT in a recent model for music demixing, to investigate the impact on the results.
\end{abstract}

\newpagefill

\renewcommand{\abstractname}{R{\'e}sum{\'e}}
\begin{abstract}
\phantomsection
\addcontentsline{toc}{section}{\abstractname}
	La s{\'e}paration des sources musicales est la t{\^a}che d'{\'e}xtraire une estimation d'une ou plusieurs sources ou instruments musicaux (p. ex. les percussions ou la voix) dans un enregistrement audio musicale. La t{\^a}che du d{\'e}mixage de musique consid{\`e}re le cas o{\`u} l'enregistrement audio musicale est s{\'e}par{\'e} en toutes ses sources constitutives qui peuvent {\^e}tre resomm{\'e}s pour recr{\'e}er l'ensemble d'origine. Les mod{\`e}les pour le d{\'e}mixage de musique qui utilisent la transformée de Fourier {\`a} court terme (TFCT) pour repr{\'e}senter les signaux musicaux sont populaires est ont obtenus du succ{\'e}s au cours des derni{\`e}res ann{\'e}es. Toutefois, la r{\'e}solution fix{\'e}e temporelle et fr{\'e}quentielle de la TFCT, r{\'e}sultant du principe d'incertitude de temps-fréquences, n{\'e}cessite un compromis entre la r{\'e}solution temporelle et fr{\'e}quentielle qui peut affecter consid{\'e}rablement les r{\'e}sultats du d{\'e}mixage de musique. La transform{\'e}e Constant-Q en tranche (sliCQT) est une transform{\'e}e temps-fr{\'e}quence avec une r{\'e}solution variable temporelle et fr{\'e}quentielle qui {\'e}vite le compromis de la TFCT. Le mod{\`e}le propos{\'e} par cette th{\`e}se remplace la TFCT par la sliCQT dans un mod{\`e}le de d{\'e}mixage de musique pour investiguer l'impact sur les r{\'e}sultats.
\end{abstract}

\end{document}
