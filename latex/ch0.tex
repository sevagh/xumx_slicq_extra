\documentclass[report.tex]{subfiles}
\begin{document}

\begin{abstract}
	Music source separation is the task of extracting an estimate of one or more isolated sources or instruments (for example, drums or vocals) from musical audio. The task of music demixing or unmixing considers the case where the musical audio is separated into an estimate of all of its constituent sources that can be summed back to the original mixture. Models for music demixing that use the Short-Time Fourier Transform (STFT) as their representation of music signals are popular and have achieved success in recent years. However, the fixed time-frequency resolution of the STFT, arising from the time-frequency uncertainty principle, requires a tradeoff in time-frequency resolution that can significantly affect music demixing results. The sliced Constant-Q Transform (sliCQT) is a time-frequency transform with varying time-frequency resolution that avoids the time-frequency tradeoff of the STFT. The model proposed by this thesis replaces the STFT with the sliCQT in a recent model for music demixing, to investigate the impact on the results.
\end{abstract}

\end{document}
