\documentclass[report.tex]{subfiles}
\begin{document}

\begin{appendices}

\section{Testbench computer specifications}
\label{appendix:computerspec}

The hardware and software specifications of the computer which produced the results shown throughout this thesis are as follows:
\begin{tight_enumerate}
	\item
		Motherboard: Gigabyte Aorus X570 Elite Wifi
	\item
		CPU: AMD Ryzen 5950X
	\item
		Memory: 64GB DDR4
	\item
		Storage: 1TB ADATA SX8200PNP NVMe
	\item
		GPU, primary: NVIDIA RTX 3080 Ti (12GB memory)
	\item
		GPU, secondary: NVIDIA RTX 2070 Super (8GB memory)
	\item
		OS: Fedora 34 Workstation Edition, 64-bit
	\item
		Linux kernel version: 5.133.10-200
	\item
		Python 3 version: 3.9.6 (default, Jul 16 2021, 00:00:00)
	\item
		NVIDIA driver version: 470.63.01
	\item
		NVIDIA CUDA toolkit version: 11.4
\end{tight_enumerate}

\newpagefill

\section{Code availability}
\label{appendix:codeavail}

The code projects associated with this thesis are published as open-source software to encourage reproducibility of results.

They are split across the following projects:
\begin{tight_enumerate}
	\item
		NSGT/sliCQT PyTorch copy from sections \ref{sec:torchslicq} and \ref{sec:improvelib}\\
		\url{https://github.com/sevagh/nsgt}
	\item
		museval (BSS metrics evaluation) CuPy copy from Section \ref{sec:fasterbsscupy}:\\
		\url{https://github.com/sevagh/sigsep-mus-eval}
	\item
		xumx-sliCQ neural network from Section \ref{sec:neuralnet}:\\
		\url{https://github.com/sevagh/xumx-sliCQ}
	\item
		LaTeX files and scripts for generating this thesis, including all plots and results:\\
		\url{https://gitlab.com/sevagh/xumx_slicq_extra}
	\item
		Submissions made to the ISMIR 2021 Music Demixing Challenge:\\
		\url{https://gitlab.aicrowd.com/sevagh/music-demixing-challenge-starter-kit}
\end{tight_enumerate}

All Python environments were designed to be reproducible with pip requirements.txt files or Conda environment files bundled with the source code:

\begin{tight_enumerate}
	\item
		Pip file for oracles, trained model evaluations, boxplot creation, and performance benchmarks:\\
		\url{https://gitlab.com/sevagh/xumx_slicq_extra/-/blob/main/mss_evaluation/mss-oracle-experiments/requirements-cupy.txt}
	\item
		Conda environment file for the NSGT/sliCQT PyTorch implementation:\\
		\url{https://github.com/sevagh/nsgt/blob/main/conda-env.yml}
	\item
		Conda environment file for the xumx-sliCQ neural network:\\
		\href{https://github.com/sevagh/xumx-sliCQ/blob/main/scripts/environment-gpu-linux-cuda11.yml}{https://github.com/sevagh/xumx-sliCQ/blob/main/scripts/environment-gpu-linux-cuda11.yml}
\end{tight_enumerate}

I take code availability and reproducibility seriously. Feel free to e-mail me\footnote{\href{mailto:sevag.hanssian@mail.mcgill.ca}{sevag.hanssian@mail.mcgill.ca}, \href{mailto:sevagh+thesis@pm.me}{sevagh+thesis@pm.me}} if you encounter any errors, discrepancies, or difficulties with reproducing any of the results.

\newpagefill

\section{Octave scale for the NSGT}
\label{appendix:octscale}

\todo[inline]{fix up the wording}

The octave scale of the NSGT library takes a bins-per-octave (bpo) argument from which it computes the total number of frequency bins, instead of the logarithmic scale which takes the total number of frequency bins as a direct argument. Equation \eqref{equation:bpo1} describes how to compute the total bins from the bins-per-octave setting of the octave scale:
\begin{align}
	K = [B \log_{2}(\sfrac{\xi_{\text{max}}}{\xi_{\text{min}}} + 1]\tag{24}\label{equation:bpo1}
\end{align}

where $K$ is the total bins, $B$ is the bins-per-octave, and $\xi_{\text{min,max}}$ are the minimum and maximum frequencies. This equation was shown previously in Section \ref{sec:cqt}. The difference between the logarithmic and octave scales is shown in Code Listing \ref{code:octvlog}. The resulting frequency scale is plotted in Figure \ref{fig:octvlog}.

\begin{figure}[h]
  \centering
 \begin{minipage}{\textwidth}
  \centering
\setlength\partopsep{-\topsep}
\begin{inputminted}[linenos,breaklines,frame=single,firstline=4,lastline=16,fontsize=\scriptsize]{text}{./scripts/fscale.py}
\end{inputminted}
 \subfloat{(a) Code snippet defining octave and log scales}
 \vspace{1em}
 \end{minipage}
 \begin{minipage}{\textwidth}
  \centering
\begin{minted}[numbersep=\mintednumbersep,linenos,mathescape=true,breaklines,frame=single,escapeinside=||,fontsize=\scriptsize]{text}
|\$ python scripts/fscale.py 3 # invoke script for bpo=3|
|bpo: 3, bins: 21, len(oct): 21, len(log): 21|
\end{minted}
 \subfloat{(b) Output printed by above}
 \end{minipage}
  \captionof{listing}{Octave and log scales for the NSGT}
  \label{code:octvlog}
\end{figure}

\begin{figure}[ht]
	\centering
	\includegraphics[width=\textwidth]{./images-freqscales/log_vs_oct.png}
	\caption{Frequency bins for the octave and log scales}
	\label{fig:octvlog}
\end{figure}


\end{appendices}

\end{document}
