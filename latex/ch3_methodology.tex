\documentclass[report.tex]{subfiles}
\begin{document}

\section{Methodology}

The proposed adaptation of Open-Unmix (\cite{umx}) to use the sliCQ transform (\cite{slicq}) was named xumx-sliCQ, and is the subject and main result of this thesis. The following methodology section will cover how xumx-sliCQ was written step-by-step.

\subsection{NSGT Python library}

The starting point for the sliCQ transform used in this paper is the Python reference library\footnote{\url{https://github.com/grrrr/nsgt}} for the NSGT (\cite{slicq}) and sliCQ transform. It was forked to my own GitHub profile,\footnote{\url{https://github.com/sevagh/nsgt}} where modifications were made to support its use in GPU deep learning, and to add some extra features such as new frequency scales.

\subsubsection{Matrix and ragged forms}

\todo[inline]{first describe the reference library's generator-based design for ragged and matrix}

\subsubsection{Overlapping slices and plotting sliCQ spectrograms}

\ichfeedback{overlap-add 50\% slice business here too, showing its non-invertible, use Essentia to support argument, justifying how i need to add a final convolutional layer to grow by 50\%}

\todo[inline]{overlap essentia stuff: \url{https://mtg.github.io/essentia-labs/news/2019/02/07/invertible-constant-q/}}

\todo[inline]{show slicq overlap and plotting code from library}

\subsubsection{Frequency scales and slice length selection}

The reference library contains the octave and log scales, which are used for creating a constant-Q transform with logarithmic frequency spacing deriving from the 12-tone Western pitch scale. It also includes the Mel psychoacoustic scale.

In my fork of the NSGT library, two additional frequency scales were added:

\begin{enumerate}
	\item
		Bark psychoacoustic scale
	\item
		Variable-Q scale (\cite{variableq1,variableq2})
\end{enumerate}

\todo[inline]{intention to support minimum slice length support, Bark scale, VQ scale}

\subsubsection{PyTorch optimizations}

\todo[inline]{explain how this is necessary for neural networks and deep learning}

\todo[inline]{describe optimizations done here}

\todo[inline]{describe optimizations case by case or line by line?}

\subsubsection{Grouping frequency bins by time resolution}

\todo[inline]{call it the ragged transform}

\subsection{Choosing sliCQ parameters with oracle estimators}

The first experiment to run is to discover which configuration of the NSGT has the potential to surpass the maximum performance of time-frequency masking using the STFT spectrogram. This should indicate whether substituting the NSGT in Open-Unmix is worthwhile.

\subsubsection{Mix-phase oracle}

Given ground truth data, such as what is available in MUSDB18, we have available to us the individually recorded sources for each track. From this, we can compute the ideal or oracle masks::

\begin{flalign}
	\nonumber \text{given: } & x_{\text{drums}}, x_{\text{vocals}}, x_{\text{bass}}, x_{\text{other}}\\
	\nonumber & x_{\text{mix}} = \sum{x}\\
	\nonumber & \hat{X} = \text{STFT}(x), \text{complex-valued}\\
	\nonumber & |\hat{X}|^{\alpha} = \text{magnitude STFT of } x \text{ raised to } \alpha \text{ power, real-valued}\\
	\nonumber & \text{Mask}_{\text{source}} = \frac{|\hat{X}|_{\text{source}}^{\alpha}}{\sum{|\hat{X}|^{\alpha}}}\\
	\nonumber & \hat{Y}_{\text{source}} = \text{Mask}_{\text{source}} * \hat{X}_{\text{mix}}\\
	\nonumber & y = \text{ISTFT}(\hat{Y})\\
	\nonumber & y = \text{ideal estimate of } x
\end{flalign}

Finally, using these pairs of $x, y$, we can get the maximum possible BSS metrics of any algorithm or model based on time-frequency masking. This is the methodology used to find the so-called oracle estimator or oracle mask, which represents the maximum possible performance of any algorithm or model for music source separation that uses spectrogram masking (CITE UMX, etc.).

Approaching or surpassing the ideal mask performance is a common benchmark in source separation literature. \todo{weak} Demucs talks about it but doesn't achieve it.

\subsubsection{sliCQ parameter search}

The sliCQ transform has several configurable parameters. Their choices and considered value ranges are described in tables \ref{table:slicqfreqparam} and \ref{table:slicqotherparam}. Parameter ranges were considered from various CQT and NSGT papers. \textcite{invertiblecqt} provides sliCQ parameter ideas for the minimum frequency and bins: they varied the minimum frequency between 10.0--130.0 Hz ($\xi_{\text{min}} \in [10.0, 130.0]$) and bins-per-octave from 12 to 192 ($B \in [12, 192]$).

Bins-per-octave applies the Western pitch scale by using multiples of 12 frequency bins, corresponding to the 12-tone scale. The total number of frequency bins from the bins-per-octave can be computed from the following formula (\cite{invertiblecqt}):
\begin{flalign}
	K = \text{integer}([B*\log_{2}(\frac{\xi_{\text{max}}}{\xi_{\text{min}}}) + 1])
\end{flalign}

The min and max of 12--192 bins-per-octave therefore corresponds to the total number of frequency bins 122--1941. In the case of other frequency scales, the total number of bins is controlled directly -- 12 Mel bins results in a 12-bin transform output (or 13 with the DC bin, 0 Hz, added). For that reason, in this thesis the octave scale with the bins-per-octave argument is not used. Instead, the logarithmic scale is used directly for output frequency bin control. The result is still the constant-Q transform, except removed from 12-tone pitch scale analogy.

\begin{table}[ht]
	\centering
\begin{tabular}{ |l|l|l| }
	 \hline
	 Parameter & Values & Notes \\
	 \hline
	 \hline
	 Frequency scale & \makecell[l]{ Constant-Q \\ Variable-Q \\ Mel \\ Bark } & \\
	 \hline
	 Minimum frequency (Hz) & 10.0 -- 130.0 & \\
	 \hline
	 Maximum frequency (Hz) & 22050.0 & Fixed to Nyquist for performance reasons \\
	 \hline
	 Frequency bins & 10 -- 300 & Constrain sliCQ transform size \\
	 \hline
	 Slice length & Auto & Minimum to support frequency scale \\
	 \hline
\end{tabular}
	\caption{Frequency scale parameters for the sliCQ transform}
	\label{table:slicqfreqparam}
\end{table}

\subsubsection{Random grid search}

\textcite{randomgrid} know what they're talking about

\subsection{Open-Unmix with the sliCQ transform}

\todo[inline]{incorporate my sparsity hypothesis/justification for nsgt}

\todo[inline]{better hope umx shines here}

\subsubsection{Working with the ragged sliCQ transform}

\ichfeedback{describe here how frequency bins are grouped by the same time-resolution, to produce a list of ``blocks'' of time-frequency coefficients}

\subsubsection{Convolutional neural network architecture}

\todo[inline]{grais and plumbley 2 papers can go here}

\subsubsection{Improved loss functions from CrossNet-Open-Unmix}

\end{document}
