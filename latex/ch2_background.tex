\documentclass[report.tex]{subfiles}
\begin{document}

\section{Background}
\label{sec:background}

\subsection{Acoustic signals and the time-domain waveform}
\label{sec:timedomain}

\textcite[Chapter~2]{discretebook} define a signal as a term which

\begin{quote}
	conveys information about the state or behavior of a physical system... [S]ignals are synthesized for the purpose of communicating information between humans or between humans and machines... Signals are represented mathematically as functions of one or more independent variables. For example, a speech signal is represented mathematically as a function of time, and a photographic image is represented as a brightness function of two spatial variables.
\end{quote}

\textcite[Chapter~2]{moore} describes the more specific case of sound, or audio signals, as follows:

\begin{quote}
	Sound originates from the vibration of an object. This vibration is impressed upon the surrounding medium (usually air) as a pattern of changes in pressure
\end{quote}

The waveform is defined as the function of pressure variation plotted against time (\cite{moore, melbook}), and is real-valued and continuous in both time and amplitude (\cite[Chapter~2]{melbook}). A continuous-time signal $x$, also called an analog signal, is denoted by $x_{a}(t)$. For digital processing, continuous-time signals need to be sampled periodically to form a sequence of numbers (\cite[Chapter~2]{discretebook}). The resultant domain is called discrete time, and a discrete-time signal is denoted by $x[n]$.  A continuous-time signal and its discrete representation are both shown in figure \ref{fig:discretecontinuous}.

\begin{figure}[ht]
	\centering
	\subfloat[Continuous-time waveform]{\includegraphics[height=2.4cm]{./images-tftheory/continuoustime.png}}\\
	\subfloat[Discrete-time waveform]{\includegraphics[height=2.4cm]{./images-tftheory/discretetime.png}}
	\caption{A continous-time signal and its discrete-time representation sampled with $T = 125\mu s$ (\cite[Chapter~2]{discretebook})}
	\label{fig:discretecontinuous}
\end{figure}

\newpagefill

Continuous-time signals are converted into discrete-time representations by two processes: \textit{sampling}, or considering periodic points on the continuous time axis throughout the waveform, and the \textit{quantization} of the amplitude values of the waveform at those points to find the closest digital number (\cite[Chapter~2]{melbook}). The sampling process is controlled by the sampling period $T \text{ seconds}$, or the sampling (or sample) rate $F_{s} = \sfrac{1}{T} \text{ Hz}$, which define the periodicity with which time values are considered. The amplitude quantization process is controlled by the number of quantization levels $2^{B}$, where $B$ is the number of bits per sample of the representation.

First, continuous time needs to be sampled into the discrete time domain. The relationship of a discrete-time signal $x[n]$ of a continuous-time signal $x_{a}(t)$ is defined by $x[n] = x_{a}(nT)$, where $n$ is an integer and $T = \sfrac{1}{F_{s}}$ is the sampling period. The Nyquist-Shannon sampling theorem (\cite[Chapter~4]{discretebook}), described independently by \textcite{nyquist1928} and \textcite{shannon1948}, states that the maximum frequency of a signal that can be represented by a sampling rate $F_{s}$ is $F_{\text{nyq}} = \sfrac{F_{s}}{2}$, which is also called the Nyquist rate or Nyquist frequency. The diagram in figure \ref{fig:aliasing} shows the phenomenon of \textit{undersampling}, which occurs when a time-domain signal is undersampled, i.e. $F_{\text{sig}} > F_{\text{nyq}}$, and the continuous-time signal cannot be reconstructed accurately (\cite[Chapter~4]{dspfirst}). We refer the readers to the chapter in the textbook for a more full description of challenges in signal sampling.

\begin{figure}[ht]
	\centering
	\includegraphics[width=0.5\textwidth]{./images-tftheory/aliasing_undersampling.png}
	\caption{An undersampled cosine wave and the resulting incorrect reconstruction (\cite{dspfirst})}
	\label{fig:aliasing}
\end{figure}

Second, the continuous amplitude needs to be quantized. Stated by \textcite[Chapter~4]{discretebook}, ``the quantizer is a nonlinear system whose purpose is to transform the input sample $x[n]$ into one of a finite set of prescribed values.'' The second variable which relates to the quantization process is the number of quantization levels. The quantization operation is represented as $ \hat{x}[n] = Q(x[n])$, where $\hat{x}[n]$ is the quantized sample. Quantization levels can be defined to be uniform (evenly spaced) or nonuniform, and in essence the sample values are rounded to the nearest quantization level $2^{B}$, recalling that $B$ is the number of bits per sample in the representation (\cite{discretebook}). An A/D or ADC (analog-to-digital converter) circuit and its operation on a waveform is shown in figure \ref{fig:adccircuit}.

\begin{figure}[ht]
	\centering
	\subfloat[Analog-to-digital converter (ADC)]{\includegraphics[height=2.2cm]{./images-tftheory/adc1.png}}
	\hspace{0.1em}
	\subfloat[ADC details: sampler and quantizer]{\includegraphics[height=2.2cm]{./images-tftheory/adc2.png}}\\
	\vspace{0.1em}
	\subfloat[Waveform results of the ADC process]{\includegraphics[width=0.85\textwidth]{./images-tftheory/adc3.png}}\\
	\caption{An ADC converter circuit showing the time sampling and amplitude quantizing operations (\cite[Chapter~4]{discretebook})}
	\label{fig:adccircuit}
\end{figure}

\newpagefill

\subsection{Transforms of acoustic signals}
\label{sec:freqdomain}

\textcite[Chapter~2B]{moore} states that

\begin{quote}
	[a]lthough all sounds can be specified by their variation in pressure with time, it is often more convenient, and more meaningful, to specify them in a different way when the sounds are complex. This method is based on a theorem by Fourier, who proved that almost any complex waveform can be analyzed, or broken down, into a series of sinusoids with specific frequencies, amplitudes, and phases. This is done using a mathematical procedure called the Fourier transform.
\end{quote}

Throughout this section, the description of the Fourier transform of acoustic signals and the evolution of further transforms that build on it will be covered in detail.

\textcite{ltfat}'s Large Time-Frequency Analysis toolbox (LTFAT) is ``a Matlab/Octave toolbox for working with time-frequency analysis and synthesis.''\footnote{\url{http://ltfat.org/}} It contains a test signal from the glockenspiel instrument, loaded by the \Verb#gspi# function.\footnote{\url{https://ltfat.github.io/doc/signals/gspi.html}} This signal is used often in audio signal processing papers on the topic of time-frequency (\cite{doerflerphd, balazs, jaillet, tfjigsaw, invertiblecqt, wmdct}); in the demo website\footnote{\url{https://homepage.univie.ac.at/monika.doerfler/StrucAudio.html}} of \textcite{wmdct}, ``the famous glockenspiel signal'' is mentioned and used for many of the examples. This is because the glockenspiel contains both tonal and transient properties, which have conflicting needs for time and frequency resolution in their analysis. Figure \ref{fig:glockwaveform} shows the discrete-time waveform of the glockenspiel signal. For the rest of this chapter, demonstrations of each transform will use this same glockenspiel signal as the input $x[n]$.

\begin{figure}[ht]
	\centering
	\includegraphics[width=0.825\textwidth]{./images-gspi/gspi_time_domain.png}
	\caption{Glockenspiel waveform}
	\label{fig:glockwaveform}
\end{figure}

The signal contains 262144 samples, and is sampled with a rate of 44100 Hz, meaning the total duration is $\sfrac{262144}{44100} \approx$ 5.94 seconds.

\newpagefill

\subsubsection{Frequency analysis and the Fourier transform}
\label{sec:freqanal}

The Fourier transform originated as an integral transform in mathematics, which are a class of ``useful tools for solving problems involving certain types of partial differential equations (PDEs), mainly when their solutions on the corresponding domains of definition are difficult to deal with'' (\cite{fourierhistory}). The Fourier transform was originally introduced by Joseph Fourier in his earlier papers (\cite{fourierhist1, fourierhist2}), and fully expanded and collected in his seminal work on heat (\cite{fourierheat}). The connection of the Fourier transform to music is described by \textcite{fouriermusic}, who state that

\begin{quote}
	[b]eyond the scope of thermal conduction, Joseph Fourier's treatise on the Analytical Theory of Heat (1822) profoundly altered our understanding of acoustic waves. It posits that any function of unit period can be decomposed into a sum of sinusoids, whose respective contribution represents some essential property of the underlying periodic phenomenon. In acoustics, such a decomposition reveals the resonant modes of a freely vibrating string.
\end{quote}

The continuous-time Fourier transform (CTFT) of a time-domain acoustic waveform is defined by the pair of equations \ref{equation:ctft} and \ref{equation:ictft} (\cite[Chapter~11]{dspfirst}):
\begin{align}
	X(j\omega) = \int_{-\infty}^{\infty}{x(t)e^{-j\omega t}\mathit{dt}} \tag{1}\label{equation:ctft} \\
	x(t) = \frac{1}{2\pi}\int_{-\infty}^{\infty}{X(j\omega)e^{j\omega t}\mathit{d\omega}} \tag{2}\label{equation:ictft}
\end{align}

$X(j\omega)$ is referred to by \textcite{dspfirst} as the frequency-domain representation of the signal $x(t)$, as the equation \ref{equation:ictft} defines the signal $x(t)$ as a ``sum of infinitely many complex-exponential signals with $X(j\omega)$ controlling the amplitude and phases of these signals.'' The continuous-time Fourier transform provides a one-to-one mapping of the time domain to the frequency domain.

As discussed in section \ref{sec:timedomain}, signals needs to be transformed from the continuous to the discrete domain via sampling to be processed digitally or computationally. The discrete-time Fourier transform (DTFT), also called the discrete Fourier transform (DFT), is derived from a sampled version of the continuous-time Fourier transform (shown previously in equations \ref{equation:ctft} and \ref{equation:ictft}), and is defined by the pair of equations \ref{equation:dtft} and \ref{equation:idtft} (\cite[Chapter~12]{melbook}):
\begin{align}
	X(e^{j\omega}) = \sum_{n = -\infty}^{\infty}{x[n]e^{-j\omega n}} \tag{3}\label{equation:dtft} \\
	x[n] = \frac{1}{2\pi}\int_{-\pi}^{\pi}{X(e^{j\omega})e^{j\omega n}\mathit{d\omega}} \tag{4}\label{equation:idtft}
\end{align}

The Fourier transform is a complex-valued function of $\omega$, which is the variable that represents the angular frequency in radians. The Fourier transform can be expressed in the rectangular form in equation \ref{equation:rect} or polar form in equation \ref{equation:polar} (\cite[Chapter~2]{discretebook}):
\begin{align}
	X(e^{j\omega}) = X_{\text{real}}(e^{j\omega}) + j X_{\text{imag}}(e^{j\omega}) \tag{5}\label{equation:rect} \\
	X(e^{j\omega}) = |X(e^{j\omega})|e^{j\measuredangle X(e^{j\omega})} \tag{6}\label{equation:polar}
\end{align}

The quantities $|X(e^{j\omega})|$ and $\measuredangle X(e^{j\omega})$ are referred to as the magnitude and phase respectively. The Fourier transform is also referred to as the spectrum, while its magnitude and phase are the magnitude and phase spectra respectively (\cite{discretebook}).

In \textcite[Chapter~9]{discretebook} it is described that in its original formulation, the algorithmic complexity of the DFT is $O(N^{2})$, i.e. the running time of the algorithm grows proportionally the size of the input signal squared (\cite{skiena}). However, starting from legendary mathematician Carl Friedrich Gauss in 1805 (\cite{gausshist}), and reaching its most famous formulation published by \textcite{cooleytukey}, a family of efficient algorithms for the computation of the DFT by computing a series of smaller DFTs, known collectively as the Fast Fourier Transform (FFT), reduced this computation time to $O(N \log{N})$. This resulted in the FFT becoming one of the most important algorithms of the 20th century (\cite{ffttopten}). Refer to figure \ref{fig:bigo} to see the differences in the expected running times of the FFT over the naive implementation.

\begin{figure}[ht]
	\centering
	\includegraphics[width=0.7\textwidth]{./images-misc/bigo.png}\\
	\caption{Visual comparison of the ``Big-O'', or worst-case running times of algorithms (taken from \url{https://www.bigocheatsheet.com/})}
	\label{fig:bigo}
\end{figure}

The number of points, or samples, of the DFT, determines the frequency resolution, also called $\mathit{df}$ or $\mathit{\Delta f}$, which is the frequency spacing between each point in the resulting spectrum (\cite{discretebook}). The frequency resolution of an $N$-point DFT is $\mathit{df} = \sfrac{F_{s}}{N}$, where $F_{s}$ is the sampling rate of the input signal. An illustration of the magnitude and phase spectra of the DFT are shown in figure \ref{fig:glockdft}, using two different lengths of DFT to show the difference in the low and high frequency resolutions.

\begin{figure}[ht]
	\centering
	\subfloat[2048-point DFT, $\mathit{df} = 21.533 Hz$]{\includegraphics[width=\textwidth]{./images-gspi/gspi_dft.png}}\\
	\subfloat[262144-point DFT, $\mathit{df} = 0.168 Hz$]{\includegraphics[width=\textwidth]{./images-gspi/gspi_dft_bigger.png}}
	\caption{DFT of the Glockenspiel waveform. Note the richer frequency information in the higher frequency resolution transform in (b)}
	\label{fig:glockdft}
\end{figure}

\newpagefill

\subsubsection{Joint time-frequency analysis -- the Gabor transform and the Short-time Fourier Transform (STFT)}
\label{sec:jointtfa}

Continuing from the discussion of the DFT in the previous section \ref{sec:freqanal}, \textcite[Chapter~10]{discretebook} state that ``often, in practical applications of sinusoidal signal models, the signal properties (amplitudes, frequencies, and phases) will change with time. For example, nonstationary signal models of this type are required to describe radar, sonar, speech, and data communication signals. A single DFT estimate is not sufficient to describe such signals...''

\textcite{gabor1946}'s seminal signal processing paper, \textit{The Theory of Communication}, introduced significant and far-reaching concepts in the time and frequency analysis of acoustic signals. Gabor quotes famed American telecommunication engineer John Carson (\cite{carsonfamous}) to describe the limitations of the Fourier transform:

\begin{quote}
	[t]he foregoing solutions [of the Fourier transform], though unquestionably mathematically correct, are somewhat difficult to reconcile with our physical intuitions and our physical concepts of such variable frequency mechanisms as, for instance, the siren
\end{quote}

According to \textcite{korpel}, ``Gabor came to the conclusion that the difficulty lay in our mutually exclusive formulations of time analysis and frequency analysis ... he suggested a new method of analyzing signals in which time and frequency play symmetrical parts.''

Let's start with a description of the time-domain unit impulse signal or sequence (\cite[Chapter~2]{melbook}) in equation \ref{equation:delta}:
\begin{flalign}\label{equation:delta}
\delta[n] = \begin{cases}
	1 \text{\hspace{1em}} n = 0\\
	0 \text{\hspace{1em}} \text{otherwise}
\end{cases}
\end{flalign}

The impulse is a useful signal, as it is the ``simplest [time-domain] sequence because it has only one nonzero value, which occurs at n = 0. The mathematical notation is that of the Kronecker delta function'' (\cite[Chapter~5]{dspfirst}). Note the Kronecker delta function is the discrete equivalent of the Dirac delta function (\cite[Chapter~2]{melbook}). Contrast this with the DFT of a signal that is only non-zero at the 0-frequency component, or the DC component, which is a ``cosine signal with zero frequency'' (\cite[Chapter~3]{dspfirst}).

Figure \ref{fig:gaborfirst} shows the unit impulse contrasted with the DC-component DFT, and these two in fact demonstrate the mutually exclusive formulations of time and frequency. The unit impulse, which has a single non-zero value in the time domain, has an infinite extent in the frequency domain. Conversely, the DC-component DFT has a single non-zero value in the frequency domain, but has an infinite extent in the time domain.

\begin{figure}[ht]
	\centering
	\includegraphics[width=7cm]{./images-tftheory/gabor13.png}
	\caption{The mutually exclusive formulations of time and frequency represented by two extremes -- unit impulse on the bottom left, non-zero DC DFT spectrum on the top right}
	\label{fig:gaborfirst}
\end{figure}

Gabor derived the principal of time-frequency uncertainty from Heisenberg's uncertainty principle in quantum physics, which is that $\sigma_{x}\sigma_{p} \ge \sfrac{h}{4\pi}$ (\cite{hallm}). In words, ``the more precisely the position [of an electron] is determined, the less precisely the momentum is known, and conversely'' (\cite{heisenberg1927}).

Gabor's time-frequency uncertainy principle is $\sigma_{t}\sigma_{f} \ge \sfrac{1}{4\pi}$, or $\Delta t\Delta f \ge 1$. In words, ``although we can carry out the analysis [of the acoustic signal] with any degree of accuracy in the time direction or frequency direction, we cannot carry it out simultaneously in both beyond a certain limit'' (\cite{gabor1946}). This is also referred to as the Gabor limit, and $\Delta t$ and $\Delta f$ are defined by \textcite{gabor1946} as ``the uncertainties inherent in the definition of the epoch t and frequency f of an oscillation.'' Gabor referred to the time-frequency tile $\Delta t \Delta f$ the \textit{logon}, or smallest possible unit of time-frequency information.

The result of the time-frequency uncertainty principle is a property of the choice of the Fourier transform to swap between the mutually exclusive domains of time and frequency. To further elaborate that this is a consequence of the Fourier transform, several psychacoustic studies have shown that humans can exhibit better time-frequency resolution than the Gabor limit. \textcite{psycho2} describes one of these experiments:

\begin{quote}
	It is concluded that models based on a place (spectral) analysis should be subject to a limitation of the type $\Delta f \cdot d \ge \text{constant}$, where $\Delta f$ is the frequency difference limen (DL) for a tone pulse of duration d. [...]  It was found that at short durations the product of $\Delta f$ and d was about one order of magnitude smaller than the minimum predicted [...]
\end{quote}

More recently, according to \textcite{psycho1}:

\begin{quote}
	[w]e have conducted the first direct psychoacoustical test of the Fourier uncertainty principle in human hearing, by measuring simultaneous temporal and frequency discrimination. Our data indicate that human subjects often beat the bound prescribed by the uncertainty theorem, by factors in excess of 10.
\end{quote}

Note that these are not a refutation of Gabor's ideas, but in fact a confirmation. \textcite{gabor1946} himself stated that ``most sound analysis and processing tools today continue to use models based on spectral theories... [w]e believe it is time to revisit this issue.'' \textcite{psycho1} further state that:

\begin{quote}
	[i]n time-frequency analysis, it has been proven that linear operators cannot exceed the uncertainty bound... Nonlinearity does not by itself confer any acuity advantage, and in fact most nonlinearities are merely distortions and thus deleterious. However [...] any carefully crafted analysis that can beat this limit must necessarily be nonlinear.
\end{quote}

Nevertheless, if we choose to proceed with time-frequency analysis despite the limitations, it's preferable to minimize time-frequency uncertainty, or to set the \textit{logon} ($\Delta t \Delta f$) to it lowest possible value, $1$. \textcite{gabor1946} states and answers the question as follows:

\begin{quote}
What is the shape of the signal for which the product $\Delta t \Delta f$ actually assumes the smallest possible value? [... it is] the modulation product of a harmonic oscillation of any frequency with a pulse of the form of the probability function
\end{quote}

\textcite{gabor1946} discovered that this can be done by multiplying overlapping, temporally consecutive portions of the input signal with shifted copies of the Gaussian window function, and by taking the Fourier transform of the Gaussian-windowed segments of the signal. In other words, Gabor proposed that any signal of finite energy can be decomposed into a linear combination of time-frequency shifts of the Gaussian function. The Gabor transform $G(f)$ of a discrete-time signal $x(n)$ is described in equation (1):
\begin{flalign}
	\nonumber \mathbf{G(f)} &= [G_{1}(f), G_{2}(f), ..., G_{k}(f)]\\
	G_{m}(f) &= \sum_{n = -\infty}^{\infty}x(n)g(n-\beta m)e^{-j2\pi \alpha n},
\end{flalign}

where $g(\cdot)$ is a Gaussian low-pass window function localized at 0, $G_{m}(f)$ is the DFT of the signal centered around time $\beta m$, and $\alpha$ and $\beta$ control the time and frequency resolution of the transform.

The mathematical details, derivations, and proofs of these equations are beyond the scope of this thesis. However, an interesting connection between \textcite{gabor1946}'s equations and \textcite{shannon1948}'s sampling theorem can be shown with a precursory overview of the equations. Gabor's original paper describes the Gaussian functions by the following set of equations:

\begin{flalign}
	\psi(t) &= \text{e}^{-\alpha^{2}(t-t_{0})}\text{cis}(2\pi f_{0}t + \phi)\\
	\phi(f) &= \text{e}^{-\frac{\pi}{\alpha}^{2}(f-f_{0})^{2}}\text{cis}[-2\pi(f - f_{0}) + \phi)]\\
	\Delta t &= \sqrt{\frac{\pi}{2}}\frac{1}{\alpha}, \Delta f = \frac{1}{\sqrt{2\pi}}\alpha
\end{flalign}

The parameter $\alpha$ of the Gaussian function determines both $\Delta t$ and $\Delta f$. Figure \ref{fig:gauswidth} shows the effect of the width of the Gaussian function on the time-frequency resolution, and figure \ref{fig:gabortf} shows different sizes of \textit{logon} in the time-frequency plane.

\begin{figure}[ht]
	\centering
	\includegraphics[width=5cm]{./images-tftheory/gabor14.png}
	\caption{The width of the Gaussian function, controlled by $\alpha$, and its effect on $\Delta t$ and $\Delta f$}
	\label{fig:gauswidth}
\end{figure}

\begin{figure}[ht]
	\centering
	\subfloat[Mutually exclusive time and frequency, and the lower bound of time-frequency resolution defined by Gabor's limit]{\includegraphics[height=3.4cm]{./images-tftheory/gabor3.png}}\\
	\subfloat[High frequency/low time, and low frequency/high time resolution logons]{\includegraphics[height=3.2cm]{./images-tftheory/gabor4.png}}
	\caption{Different time-frequency resolutions (\cite{gabordiagrams})}
	\label{fig:gabortf}
\end{figure}

\newpagefill

Alternate forms of the Gabor equations are derived by \textcite{gabor2}:
\begin{flalign}
	C_{\mathit{jk}}(t) &= e^{\frac{-\pi (t - j\Delta t)^{2}}{\alpha^{2}}} \cos{[2\pi k \Delta f(t-j\Delta t)]}\\
	S_{\mathit{jk}}(t) &= e^{\frac{-\pi (t - j\Delta t)^{2}}{\alpha^{2}}} \sin{[2\pi k \Delta f(t-j\Delta t)]}\\
	\phi_{\mathit{jk}} &= C_{\mathit{jk}} + jS_{\mathit{jk}}
\end{flalign}

As $\alpha \rightarrow \infty$, this reduces to the Fourier transform:
\begin{flalign}
	C_{\mathit{jk}}(t) &= \cos{[2\pi j k \Delta f(t - j \Delta t)]}\\
	S_{\mathit{jk}}(t) &= \sin{[2\pi j k \Delta f(t - j \Delta t)]}\\
	\phi_{\mathit{jk}}(t) &= e^{[2\pi j k \Delta f(t - j \Delta t)]}
\end{flalign}

As $\alpha \rightarrow 0$, this reduces to Dirac delta functions spaced at $\Delta t$, or unit impulse function as shown previously in \ref{equation:delta}:
\begin{flalign}\label{equation:twodelta}
	C_{\mathit{jk}}(t) &= S_{\mathit{jk}}(t) = \delta (t - j\Delta t) \\
	\phi_{\mathit{jk}}(t) &= \delta (t - j \Delta t) + j \delta (t - j \Delta t) = a_{\mathit{ij}} + j b_{\mathit{ij}} 
\end{flalign}

We now explore the connection to Shannon's sampling theorem (\cite{shannon1948}). Shannon states that ``to reconstruct $\psi$ we must take equally spaced samples at a minimum of the Nyquist frequency, which is twice the maximal frequency.'' In equation \ref{equation:twodelta}, the $\alpha = 0$ limit of the Gabor equations represents two samples, $a_{\mathit{ij}}$ and $b_{\mathit{ij}}$, for each $\Delta t$ interval.

The STFT, or short-time Fourier transform, has been described independently from Gabor's work (\cite{stftindie}), but additional research in the 1980s (\cite{dictionary}) led to the STFT being formalized and described as a special case of the Gabor transform, in recognition of Gabor's pioneering work. The STFT $X(f)$ of a discrete-time signal $x(n)$ is described in equation (2):
\begin{flalign}
	\nonumber \mathbf{X(f)} &= [X_{1}(f), X_{2}(f), ..., X_{k}(f)]\\
	X_{m}(f) &= \sum_{n = -\infty}^{\infty}x(n)g(n-mR)e^{-j2\pi f n},
\end{flalign}

where $g(\cdot)$ are the time-shifted, localized windows, $X_{m}(f)$ is the DFT of the audio signal centered about time $mR$, and $R$ is the hop size between successive time-shifts of the window. Note how similar equations (1) and (2) are, which is expected since the original Gabor transform is the STFT with a Gaussian window. Practically, the STFT allows the use of different windows and overlap sizes (\cite{stftinvertible}), as long as overlap-add conditions are respected.\footnote{\url{https://www.mathworks.com/help/signal/ref/iscola.html}}

\textcite{doerflersouls} describe the powerful properties of the STFT and why its such an important tool in the analysis of acoustic signals:

\begin{quote}
	... the STFT has at least three souls: it is the Fourier transform of windowed portions of the signal, it is the convolution of the signal with modulated versions of the window and it is the scalar product of the signal with time-shifted modulated versions of the window. These three souls can be exploited in the applications, for the computation of the STFT and for its sampling ...
\end{quote}

Figure \ref{fig:stfts} shows the Gabor transform alongside the STFT using a popular choice of window, the Hamming window, for three different window sizes: 128 samples, 2048 samples, and 16384 samples, which at the sample rate of the glockenspiel signal (44100 Hz) represent $\sfrac{128}{44100}\cdot 1000 = $ 2.9 ms, 46.44 ms, and 371.52 ms respectively. The Hamming window was chosen because it is the default window in the MATLAB \Verb#spectrogram# function.\footnote{\url{https://www.mathworks.com/help/signal/ref/spectrogram.html}}

The Hamming window is advantageous over the Gaussian window because it is exactly zero outside of its specified range. The Gaussian window asymptotially approaches zero but never reaches it. \textcite{gabor2} describes that the Gabor functions are not strictly local, because from their infinite Gaussian envelope, they stretch out to infinity. This is biologically problematic, but the Gaussian envelope is well-localized (99.7\% of its area is within 3 standard deviations of the mean), so that the Gabor transform with Gaussian windows is in practice a ``good enough approximation'' of biology. The Gaussian and Hamming windows are contrasted in figure \ref{fig:windows}.

\begin{figure}[ht]
	\centering
	\includegraphics[width=10cm]{./images-tftheory/gaussianvshann.png}
	\caption{The Gaussian and Hamming windows}
	\label{fig:windows}
\end{figure}

\begin{figure}[ht]
	\centering
	\subfloat[STFT, 128-sample Hamming window]{\includegraphics[width=0.5\textwidth]{./images-gspi/gspi_hamm_128.png}}
	\subfloat[Gabor transform, 128-sample Gaussian window]{\includegraphics[width=0.5\textwidth]{./images-gspi/gspi_gauss_128.png}}\\
	\subfloat[STFT, 2048-sample Hamming window]{\includegraphics[width=0.5\textwidth]{./images-gspi/gspi_hamm_2048.png}}
	\subfloat[Gabor transform, 2048-sample Gaussian window]{\includegraphics[width=0.5\textwidth]{./images-gspi/gspi_gauss_2048.png}}\\
	\subfloat[STFT, 16384-sample Hamming window]{\includegraphics[width=0.5\textwidth]{./images-gspi/gspi_hamm_16384.png}}
	\subfloat[Gabor transform, 16384-sample Gaussian window]{\includegraphics[width=0.5\textwidth]{./images-gspi/gspi_gauss_16384.png}}
	\caption{Visual comparison of the Hamming-window STFT and the Gaussian-window STFT (i.e. the Gabor transform), using the Glockenspiel signal. Note how the time resolution decreases with the window size (blurry temporal events i.e. vertical lines), while the frequency resolution increases (sharper frequencies i.e. horizontal lines)}
	\label{fig:stfts}
\end{figure}

\newpagefill

\subsubsection{Constant-Q Transform (CQT)}

Summary of the most relevant CQT implementations:

\begin{enumerate}
	\item
	    Brown, 1991, first proposed CQT with a naive very slow implementation.
    \item
	    Brown and Puckette, 1992: implemented a faster CQT based on a sparse representation in the frequency domain. This is the current Essentia implementation!
    \item
	    Sch{\"o}rkhuber and Klapuri, 2010: a faster CQT based on the same principles as Brown and Puckette, 1992. For first time, it is introduced an algorithm for an approximated reconstruction of the CQT coefficients. Code: \url{http://www.iem.at/~schoerkhuber/cqt2010/} - this is the librosa implementation
    \item
	    Velasco, Holighaus, D{\"o}rfler and Grill, 2011: they approach the problem differently - by means of a nonstationary Gabor transform. This allows perfect reconstruction for first time while the transform is still computationally efficient (faster than Sch{\"o}rkhuber and Klapuri, 2010). However, it does not allow real-time implementations and phases are not accurate. Code: \url{http://www.univie.ac.at/nonstatgab/toolbox.php}
    \item
	    Holighaus, D{\"o}rfler, Velasco and Grill, 2012: Based on the Velasco, Holighaus, Dörfler and Grill, 2011 - allowing perfect reconstruction. They propose sliCQT (slicing by using an overlapping window) to allow real-time computations. Code: \url{http://www.univie.ac.at/nonstatgab/toolbox.php}
    \item
	    Sch{\"o}rkhuber, Klapuri, Holighaus and Dörfler, 2014: Based on the Velasco, Holighaus, Dörfler and Grill, 2011 - allowing perfect reconstruction. They solve the problem with phases by means of a frequency mapping and they also propose a Variable-Q transform (that allows ie. ERBlets). Code: \url{http://www.cs.tut.fi/sgn/arg/CQT/}
    \item
	    Sch{\"o}rkhuber, Klapuri, Holighaus and D{\"o}rfler, 2014, a nonstationary Gabor transform, allows perfect reconstruction while the phases are still accurate. It might be interesting to implement this in Essentia.
\end{enumerate}

\todo[inline]{the original judith brown CQT is the origin of the research that led to the NSGT}

The Constant-Q transform (CQT) is time-frequency transform for musical signals, originally designed by \textcite{jbrown}, the relationship between the fundamental frequency and its harmonics on a logarithmic frequency scale more clearly than the linear frequency scale of the traditional discrete Fourier transform (DFT).

The original CQT had no inverse transform, but later works led to approximate inverses \cite{klapuricqt, fitzgeraldcqt}. 
, which has an important application in the perfectly-invertible CQT, or CQ-NSGT \cite{invertiblecqt}

The more general NSGT should be studied instead the CQT for the following reasons:
\begin{itemize}
	\item
		It solves the earlier CQT's \cite{jbrown, klapuricqt, fitzgeraldcqt} lack of stable inverse, which was a known weakness \cite{lackinverse}
	\item
		It can use other potentially interesting frequency scales besides the constant-Q logarithmic scale, such as the psychoacoustically-motivated mel, Bark, or ERB scales \todo{cite me}, or variable-Q scales \todo{cite gamma and other}
\end{itemize}

 The CQT has a high temporal resolution at high frequencies \cite{cqtransient} .

 A demonstration of the CQT is shown in figure \ref{fig:earlycqt}.

\begin{figure}[ht]
	\centering
	\subfloat[Linear frequency spectrum]{\includegraphics[height=4.75cm]{./images-tftheory/violindft.png}}
	\subfloat[Constant-Q transform]{\includegraphics[height=4.75cm]{./images-tftheory/violincqt.png}}
	\caption{Violin playing the diatonic scale, $G_{3} \text{(196Hz)} - G_{5} \text{(784Hz)}$}
	\label{fig:earlycqt}
\end{figure}

Additionally, the constant-Q transform \cite{jbrown, klapuricqt, invertiblecqt} even before its formulation as a specialized variant of the nonstationary Gabor transform \cite{balazs}, is an STFT applied with window of different sizes, which are of long duration at low frequencies to create a fine frequency resolution (and sacrificing time resolution as per the time-frequency uncertainty principle), and gradually decrease the windows in duration to improve the time resolution (and sacrifice frequency resolution). At the same time, consider that the iterative harmonic-percussive source separation algorithms in \cite{driedger, fitzgerald2} use two-pass spectral masking with two different configurations of spectrograms -- one with a large window size (4096 samples in \cite{driedger}, 16384 samples in \cite{fitzgerald2}) for representing the harmonic or pitched instruments sharply and estimating the harmonic mask, and one with a short window size (256 samples in \cite{driedger}, 1024 samples in \cite{fitzgerald2}) for representing percussion or transients more sharply and estimating the percussive mask.

The connection to the CQT, or NSGT, is that these contain within a single transform the high frequency resolution of a the large-size spectrogram in the low frequency regions, and the high time resolution of the small-size spectrogram in the high frequency regions. According to \textcite{musicsepgood}'s survey on music source separation, most spectral masking techniques try to exploit the 

\begin{figure}[ht]
	\centering
	\includegraphics[width=9cm]{./images-tftheory/tf_tradeoff_dorfler.png}
	\caption{Time-frequency tradeoff for a glockenspiel signal}
	\label{fig:dorflertradeoff}
\end{figure}

The time-frequency tradeoff is demonstrated on a musical glockenspiel signal in figure \ref{fig:dorflertradeoff}. Notice how the wide window spectrogram shows frequency components (horizontal lines) with a sharper definition than the blurry lines in the narrow window spectrogram, while the narrow window spectrogram shows temporal events (vertical lines) with a sharper definition than the wide window spectrogram.

\subsubsection{Nonstationary Gabor Transform (NSGT) and the sliCQ transform}
\label{sec:theorynsgt}

\todo[inline]{brief intro to frame theory and math stuff}

\todo[inline]{irregular time and frequency sampling, show grids, varying time-frequency resolution etc.}

\todo[inline]{arbitrary f scales and time scales}

\todo[inline]{slicq is the realtime variant}

\todo[inline]{whats the output, what does it mean, how does it relate to the FFT coefficients, time-frequency matrix}

\newpagefill

\subsection{Nonlinear frequency scales for music analysis}
\label{sec:freqscales}

\ichfeedback{better title? this is a bit long. ``Frequency scales that may be useful for music or psychoacoustic purposes'' seems like a mouthful - ``Frequency scales for music analysis``?}

\subsubsection{Scales based on Western pitch}

constant-q, log, western pitch scale, octave

variable-q - same with gamma offset

from \cite{variableq1, variableq2}, same as cq-log but with a gamma parameter

\subsubsection{Psychoacoustic scales}

mel, bark

\newpagefill

\subsection{Machine learning}
\label{sec:ml}

\ichfeedback{if i'm presenting a neural network, it's probably necessary to have this section?}

\subsubsection{Deep learning}
\label{sec:dl}

\newpagefill

\subsection{Music source separation}
\label{sec:musicsep}

\subsubsection{Task motivation and definition}

\todo[inline]{purposes and uses - why do we want to do this}

\subsubsection{Public datasets}

The most popular music stem dataset used by SISEC and SigSep is the MUSDB18 dataset (\cite{musdb18}), and more recently the HQ (high-quality) version (\cite{musdb18hq}). MUSDB18-HQ contains stereo wav files sampled at 44100 Hz representing stems (drum, vocal, bass, and other) from a collection of permissively licensed music, specifically intended for recording, mastering, mixing (and in this case, ``de-mixing'', or source separation) research. It combines earlier mixing/demixing datasets (\cite{otherdataset1, otherdataset2}).

The songs in the MUSDB18-HQ dataset have a fixed train, validation, and test split. Following the rules defined in the ISMIR 2021 Music Demixing Challenge,\footnote{\url{https://www.aicrowd.com/challenges/music-demixing-challenge-ismir-2021}} for a network to be considered trained only on MUSDB18-HQ, the predefined data splits must be used.

\subsubsection{Evaluation measures}

The SigSep\footnote{\url{https://sigsep.github.io/}} community, borrowing from the methodology of Signal Separation Evaluation Campaign (SISEC), uses the BSS (Blind Source Separation) Eval \cite{bss} objective measure for separation quality. There are 4 distinct metrics that comprise BSS:

\begin{itemize}
\item
	\textbf{ISR:} source Image to Spatial distortion Ratio
\item
	\textbf{SIR:} Signal to Interference Ratio
\item
	\textbf{SAR:} Signal to Artifacts Ratio
\item
	\textbf{SDR:} Signal to Distortion Ratio
\end{itemize}

Out of these 4 scores, SDR is the single global score which is commonly used to summarize the overall performance of a music demixing system (\cite{sdruseful}). The SDR as it was defined in the ISMIR 2021 Music Demixing Challenge (and used to rank the participants) can be computed from the following equation:

\[ \]

In the SigSep community and in the most recent SiSec evaluation (\cite{sisec2018}), the BSS evaluation measure used is BSS v4, a variant of BSS available in their Python libraries museval\footnote{\url{https://github.com/sigsep/sigsep-mus-eval}} and bsseval.\footnote{\url{https://github.com/sigsep/bsseval}} The differences between BSS as used in SiSec 2016 (\cite{sisec2016}) and BSS v4 are outlined in the bsseval project's GitHub README file:

\begin{quote}
	One particularity of BSSEval is to compute the metrics after optimally matching the estimates to the true sources through linear distortion filters. This allows the criteria to be robust to some linear mismatches... this matching is the reason for most of the computation cost of BSSEval...

	For this package, we enabled the option of having time invariant distortion filters, instead of necessarily taking them as varying over time as done in the previous versions of BSSEval. First, enabling this option significantly reduces the computational cost for evaluation because matching needs to be done only once for the whole signal. Second, it introduces much more dynamics in the evaluation, because time-varying matching filters turn out to over-estimate performance. Third, this makes matching more robust, because true sources are not silent throughout the whole recording, while they often were for short windows
\end{quote}

\subsubsection{Survey of computational approaches}

\ichfeedback{spectral masking, NMF, machine learning, deep learning - i can lean on the machine learning introduction section right before}

\todo[inline]{summary of approaches over the year e.g. nonnegative matrix factorization to machine learning to deep learning}

\subsubsection{Time-frequency masking and oracle estimators}

\ichfeedback{i think the idea of the oracle mask computed from ground truths is important enough to be in the section title}

\ichfeedback{it will come up later in the thesis when choosing hyperparameters for the sliCQ}

\subsubsection{Open-Unmix (UMX) and CrossNet-Open-Unmix (X-UMX)}

\chaptertodo{
	\url{https://papers.nips.cc/paper/2014/file/a14ac55a4f27472c5d894ec1c3c743d2-Paper.pdf}\\
	use my RNN slides from MIR-presentations\\
	UMX bi-LSTM comes from \url{https://www.researchgate.net/publication/335688695_Open-Unmix_-_A_Reference_Implementation_for_Music_Source_Separation}\\
	more umx \url{https://hal.inria.fr/hal-02293689/document}\\
	bi-LSTM here which comes from \url{https://www.researchgate.net/publication/315100151_Improving_music_source_separation_based_on_deep_neural_networks_through_data_augmentation_and_network_blending}\\
	more here \url{https://ieeexplore.ieee.org/document/7952158}\\
	more here \url{https://www.researchgate.net/profile/Yuki-Mitsufuji/publication/315100151_Improving_music_source_separation_based_on_deep_neural_networks_through_data_augmentation_and_network_blending/links/59ed4f844585151983ccdcba/Improving-music-source-separation-based-on-deep-neural-networks-through-data-augmentation-and-network-blending.pdf}\\
	finally comes from here \url{https://www.semanticscholar.org/paper/Framewise-phoneme-classification-with-bidirectional-Graves-Schmidhuber/2f83f6e1afadf0963153974968af6b8342775d82}

}

 \textcite{umx}'s deep learning model for music source separation is intended to be a near state-of-the-art, open implementation based on the open MUSDB18 and MUSDB18-HQ datasets and designed to foster source separation research \cite{musdb18, musdb18hq}. A deep neural network is used to estimate the magnitude spectrograms of the sources given a mixed song as an input. The sources are the same as the four stems per track in MUSDB18: drums, vocals, bass, other. Finally, the estimate is used to compute a soft mask.

 \subsubsection{Convolutional denoising autoencoders}

 While the discussed Open-Unmix model is based on a sequence-to-sequence LSTM architecture, a different class of neural network called convolutional autoencoders, or convolutional denoising autoencoders (CDAE), have been seeing increasing use in music demixing (\cite{plumbley1, plumbley2}).

 In particular, the CDAE networks of \textcite{plumbley1} and \textcite{plumbley2} give some simple and adaptable ideas for performing music demixing with 2D convolutional layers applied on time-frequency transforms.

 \todo[inline]{plumbley's papers here and a bit of CNN theory}

\end{document}
