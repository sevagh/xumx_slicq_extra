\documentclass[report.tex]{subfiles}
\begin{document}

\section{Background}
\label{sec:background}

\subsection{Acoustic signals and the time-domain waveform}
\label{sec:timedomain}

\textcite[Chapter~2]{discretebook} define a signal as a term which

\begin{quote}
	conveys information about the state or behavior of a physical system, and often, signals are synthesized for the purpose of communicating information between humans or between humans and machines... Signals are represented mathematically as functions of one or more independent variables. For example, a speech signal is represented mathematically as a function of time, and a photographic image is represented as a brightness function of two spatial variables.
\end{quote}

As audio or sound waves are the type of signal studied in this thesis, \textcite[Chapter~2]{moore} describes the phenomenon of sound waves as follows:

\begin{quote}
	Sound originates from the vibration of an object. This vibration is impressed upon the surrounding medium (usually air) as a pattern of changes in pressure
\end{quote}

The waveform is defined as the function of pressure variation plotted against time (\cite{moore, melbook}), and the waveform is real-valued and continuous in both time and amplitude (\cite[Chapter~2]{melbook}). A continuous-time signal $x$, also called an analog signal, is denoted by $x_{a}(t)$. For digital processing, continuous-time signals need to be sampled periodically to form a sequence of numbers (\cite[Chapter~2]{discretebook}). The resultant domain is called discrete time, and a discrete-time signal is denoted by $x[n]$.  A continuous-time signal and its discrete representation are both shown in figure \ref{fig:discretecontinuous}.

\begin{figure}[ht]
	\centering
	\subfloat[Continuous-time waveform]{\includegraphics[height=2.4cm]{./images-tftheory/continuoustime.png}}\\
	\subfloat[Discrete-time waveform]{\includegraphics[height=2.4cm]{./images-tftheory/discretetime.png}}
	\caption{A continous-time signal and its discrete-time representation sampled with $T = 125\mu s$ (\cite[Chapter~2]{discretebook})}
	\label{fig:discretecontinuous}
\end{figure}

Continuous-time signals are continuous in both time and amplitude. They are captured into discrete-time representations by \textit{sampling} the continuous time values and \textit{quantizing} the amplitude values (\cite[Chapter~2]{melbook}). The relationship of a discrete-time signal $x[n]$ of a continuous-time signal $x(t)$ is defined by the sampling period $T$:
\[ x[n] = x_{a}(nT) \]

Two important variables describe the nature of the discrete representation: the sampling rate $F_{s} = \frac{1}{T}$, and the number of quantization levels $2^{B}$, where $B$ is the number of bits per sample of the representation.

First, continuous time needs to be sampled into the discrete time domain. The sampling rate, $F_{s}$, relates to the Nyquist-Shannon sampling theorem (\cite[Chapter~4]{discretebook}), which was described by both \textcite{nyquist1928} and \textcite{shannon1948}. It states that the maximum frequency of a signal that can be represented by a sampling rate $F_{s}$ is $F_{\text{nyq}} = \frac{F_{s}}{2}$, which is also called the Nyquist rate or Nyquist frequency. The diagram in figure \ref{fig:aliasing} shows the phenomenon of \textit{undersampling}, which occurs when a time-domain signal is undersampled, i.e. the maximum frequency of the signal exceeds $F_{\text{nyq}}$, and the continuous-time signal cannot be reconstructed accurately (\cite[Chapter~4]{dspfirst}). We refer the readers to the chapter in the textbook for a more full description of challenges in signal sampling.

\begin{figure}[ht]
	\centering
	\includegraphics[width=0.5\textwidth]{./images-tftheory/aliasing_undersampling.png}
	\caption{An undersampled cosine wave and the resulting incorrect reconstruction (\cite{dspfirst})}
	\label{fig:aliasing}
\end{figure}

Second, the continuous amplitude needs to be quantized. Stated by \textcite[Chapter~4]{discretebook}, ``the quantizer is a nonlinear system whose purpose is to transform the input sample $x[n]$ into one of a finite set of prescribed values.'' The second variable which relates to the quantization process is the number of quantization levels. The quantization operation is represented as
\[ \hat{x}[n] = Q(x[n]) \]

, where $\hat{x}[n]$ is the quantized sample. Quantization levels can be defined to be uniform (evenly spaced) or nonuniform, and in essence the sample values are rounded to the nearest quantization level $2^{B}$, recalling that $B$ is the number of bits per sample in the representation (\cite{discretebook}). An A/D or ADC (analog-to-digital converter) circuit and its operation on a waveform is shown in figure \ref{fig:adccircuit}.

\begin{figure}[ht]
	\centering
	\subfloat[Analog-to-digital converter (ADC)]{\includegraphics[height=2.2cm]{./images-tftheory/adc1.png}}
	\hspace{0.1em}
	\subfloat[ADC details: sampler and quantizer]{\includegraphics[height=2.2cm]{./images-tftheory/adc2.png}}\\
	\subfloat[Waveform results of the ADC process]{\includegraphics[width=0.75\textwidth]{./images-tftheory/adc3.png}}\\
	\caption{An ADC converter circuit showing the time sampling and amplitude quantizing operations (\cite[Chapter~4]{discretebook})}
	\label{fig:adccircuit}
\end{figure}

\vfill
\clearpage

\subsection{Transforms of acoustic signals}
\label{sec:freqdomain}

\textcite[Chapter~2B]{moore} states that

\begin{quote}
	[a]lthough all sounds can be specified by their variation in pressure with time, it is often more convenient, and more meaningful, to specify them in a different way when the sounds are complex. This method is based on a theorem by Fourier, who proved that almost any complex waveform can be analyzed, or broken down, into a series of sinusoids with specific frequencies, amplitudes, and phases. This is done using a mathematical procedure called the Fourier transform.
\end{quote}

Throughout this section, the description of the Fourier transform of acoustic signals and the evolution of further transforms that build on it will be covered in detail.

\textcite{ltfat}'s Large Time-Frequency Analysis toolbox (LTFAT) is ``a Matlab/Octave toolbox for working with time-frequency analysis and synthesis.''\footnote{\url{http://ltfat.org/}} It contains a test signal from the glockenspiel instrument, loaded by the \Verb#gspi# function.\footnote{\url{https://ltfat.github.io/doc/signals/gspi.html}} This signal is used often in audio signal processing papers on the topic of time-frequency (\cite{doerflerphd, balazs, jaillet, tfjigsaw, invertiblecqt, wmdct}); in the demo website\footnote{\url{https://homepage.univie.ac.at/monika.doerfler/StrucAudio.html}} of \textcite{wmdct}, ``the famous glockenspiel signal'' is mentioned and used for many of the examples. This is because the glockenspiel contains both tonal and transient properties, which have conflicting needs for time and frequency resolution in their analysis. Figure \ref{fig:glockwaveform} shows the discrete-time waveform of the glockenspiel signal. For the rest of this chapter, demonstrations of each transform will use this same glockenspiel signal as the input $x[n]$.

\begin{figure}[ht]
	\centering
	\includegraphics[width=0.825\textwidth]{./images-gspi/gspi_time_domain.png}
	\caption{Glockenspiel waveform}
	\label{fig:glockwaveform}
\end{figure}

\vfill
\clearpage

\subsubsection{Frequency analysis and the Fourier transform}

The Fourier transform originated as an integral transform in mathematics, which are a class of ``useful tools for solving problems involving certain types of partial differential equations (PDEs), mainly when their solutions on the corresponding domains of definition are difficult to deal with'' (\cite{fourierhistory}). The Fourier transform was originally introduced by Joseph Fourier in his earlier papers (\cite{fourierhist1, fourierhist2}), and fully expanded and collected in his seminal work on heat (\cite{fourierheat}). The connection of the Fourier transform to music is described in \textcite{fouriermusic}, who state that

\begin{quote}
	[b]eyond the scope of thermal conduction, Joseph Fourier's treatise on the Analytical Theory of Heat (1822) profoundly altered our understanding of acoustic waves. It posits that any function of unit period can be decomposed into a sum of sinusoids, whose respective contribution represents some essential property of the underlying periodic phenomenon. In acoustics, such a decomposition reveals the resonant modes of a freely vibrating string.
\end{quote}

The continuous-time Fourier transform (CTFT) of a time-domain acoustic waveform is defined by the pair of equations \ref{equation:ctft} and \ref{equation:ictft} (\cite[Chapter~11]{dspfirst}):
\begin{align}
	X(j\omega) = \int_{-\infty}^{\infty}{x(t)e^{-j\omega t}\mathit{dt}} \tag{1}\label{equation:ctft} \\
	x(t) = \frac{1}{2\pi}\int_{-\infty}^{\infty}{X(j\omega)e^{j\omega t}\mathit{d\omega}} \tag{2}\label{equation:ictft}
\end{align}

$X(j\omega)$ is referred to by \textcite{dspfirst} as the frequency-domain representation of the signal $x(t)$, as the equation \ref{equation:ictft} defines the signal $x(t)$ as a ``sum of infinitely many complex-exponential signals with $X(j\omega)$ controlling the amplitude and phases of these signals.'' $x(t)$, as has been mentioned in section \ref{sec:timedomain}, is the time-domain representation of the signal -- therefore, the continuous-time Fourier transform provides a one-to-one mapping of the time domain to the frequency domain (\cite{dspfirst}).

As discussed in section \ref{sec:timedomain}, signals needs to be transformed from the continuous to the discrete domain via sampling to be processed digitally or computationally. The discrete-time Fourier transform (DTFT), also called the discrete Fourier transform (DFT), is derived from a sampled version of the continuous-time Fourier transform (shown previously in equations \ref{equation:ctft} and \ref{equation:ictft}), and is defined by the pair of equations \ref{equation:dtft} and \ref{equation:idtft} (\cite[Chapter~12]{melbook}):

\begin{align}
	X(e^{j\omega}) = \sum_{n = -\infty}^{\infty}{x[n]e^{-j\omega n}} \tag{3}\label{equation:dtft} \\
	x[n] = \frac{1}{2\pi}\int_{-\pi}^{\pi}{X(e^{j\omega})e^{j\omega n}\mathit{d\omega}} \tag{4}\label{equation:idtft}
\end{align}

\chaptertodo{
DTFT is complex - magnitude + phase + related spectral plot of glockenspiel\\
continue FFT stuff, cooley-tukey bluestein and quote its importance and O(nlogn)\\
}

Its most common implementation choice \todo{citeme}, the Fast Fourier Transform (FFT), is considered one of the most important algorithms in modern computing \todo{citeme}.

\subsubsection{Joint time-frequency analysis -- the Gabor transform and the Short-time Fourier Transform (STFT)}

\chaptertodo{
mutually exclusive formulations of time and frequency, gabor stuff\\
intersperse TF-presentation on Gabor in Chapter 2 transforms section\\
Gabor transform, STFT need diagrams (similar to TF presentations)\\
3 souls of the STFT: \url{https://www.researchgate.net/publication/307653687_Arbitrary_Phase_Vocoders_by_means_of_Warping}\\
}

\textcite{gabor1946}'s seminal signal processing paper, \textit{The Theory of Communication}, contained the first suggestion of joint time-frequency decomposition of a signal by applying the Fourier transform locally to overlapping portions of the signal multiplied by Gaussian windows. In other words, Gabor proposed that any signal of finite energy can be decomposed into a linear combination of time-frequency shifts of the Gaussian function. The Gabor transform $G(f)$ of a discrete-time signal $x(n)$ is described in equation (1):
\begin{flalign}
	\nonumber \mathbf{G(f)} &= [G_{1}(f), G_{2}(f), ..., G_{k}(f)]\\
	G_{m}(f) &= \sum_{n = -\infty}^{\infty}x(n)g(n-\beta m)e^{-j2\pi \alpha n},
\end{flalign}

where $g(\cdot)$ is a Gaussian low-pass window function localized at 0, $G_{m}(f)$ is the DFT of the signal centered around time $\beta m$, and $\alpha$ and $\beta$ control the time and frequency resolution of the transform. With his transform, Gabor also introduced the first formulation of the time-frequency uncertainty principle (which is minimized by using the Gaussian function as a window), stating that ``although we can carry out the analysis [of the acoustic signal] with any degree of accuracy in the time direction or frequency direction, we cannot carry it out simultaneously in both beyond a certain limit.'' Gabor called named the time-frequency tile the \textit{logon}, or smallest possible unit of time-frequency information. Mathematically, this can be stated as:
\[ \Delta t\Delta f \ge 1 \]

$\Delta t$ and $\Delta f$ are, as defined by Gabor, ``the uncertainties inherent in the definition of the epoch $t$ and frequency $f$ of an oscillation.'' The TF uncertainty principle arises from the fact that time and frequency are, in quantum physics terms, conjugate variables, or Fourier transforms of each other. This is further illustrated in figure \ref{fig:gabortf}, which shows the tiling of the time-frequency plane, and how frequency and time resolution must be sacrificed for one another by the lower bound of the time-frequency tile area.

\begin{figure}[ht]
	\centering
	\subfloat{\includegraphics[height=3cm]{./images-tftheory/gabor3.png}}
	\hspace{0.1em}
	\subfloat{\includegraphics[height=2.56cm]{./images-tftheory/gabor4.png}}
	\caption{A demonstration of the mutually exclusive formulations of time analysis and frequency analysis, and the lower bound of time-frequency resolution defined by Gabor's TF uncertainty principle (\cite{gabordiagrams})}
	\label{fig:gabortf}
\end{figure}

Gabor noted the need for variable-frequency analysis, stating that ``the foregoing solutions [of the Fourier transform], though unquestionably mathematically correct, are somewhat difficult to reconcile with our physical intuitions and our physical concepts of such variable frequency mechanisms as, for instance, the siren.'' Similarly, psychoacoustics research shows that humans have been able to beat the time-frequency uncertainty principle (\cite{psycho1, psycho2}), indicating the presence of nonlinear operators in the auditory system.

The STFT, or short-time Fourier transform, has been described independently from Gabor's work (\cite{stftindie}), but additional research in the 1980s (\cite{dictionary}) led to the STFT being formalized and described as a special case of the Gabor transform, in recognition of Gabor's pioneering work. The STFT $X(f)$ of a discrete-time signal $x(n)$ is described in equation (2):
\begin{flalign}
	\nonumber \mathbf{X(f)} &= [X_{1}(f), X_{2}(f), ..., X_{k}(f)]\\
	X_{m}(f) &= \sum_{n = -\infty}^{\infty}x(n)g(n-mR)e^{-j2\pi f n},
\end{flalign}

where $g(\cdot)$ are the time-shifted, localized windows, $X_{m}(f)$ is the DFT of the audio signal centered about time $mR$, and $R$ is the hop size between successive time-shifts of the window. Note how similar equations (1) and (2) are, which is expected since the original Gabor transform is the STFT with a Gaussian window. Practically, the STFT allows the use of different windows and overlap sizes (\cite{stftinvertible}), as long as overlap-add conditions are respected.\footnote{\url{https://www.mathworks.com/help/signal/ref/iscola.html}}

\subsubsection{Constant-Q Transform (CQT)}

Summary of the most relevant CQT implementations:

\begin{enumerate}
	\item
	    Brown, 1991, first proposed CQT with a naive very slow implementation.
    \item
	    Brown and Puckette, 1992: implemented a faster CQT based on a sparse representation in the frequency domain. This is the current Essentia implementation!
    \item
	    Sch{\"o}rkhuber and Klapuri, 2010: a faster CQT based on the same principles as Brown and Puckette, 1992. For first time, it is introduced an algorithm for an approximated reconstruction of the CQT coefficients. Code: \url{http://www.iem.at/~schoerkhuber/cqt2010/} - this is the librosa implementation
    \item
	    Velasco, Holighaus, D{\"o}rfler and Grill, 2011: they approach the problem differently - by means of a nonstationary Gabor transform. This allows perfect reconstruction for first time while the transform is still computationally efficient (faster than Sch{\"o}rkhuber and Klapuri, 2010). However, it does not allow real-time implementations and phases are not accurate. Code: \url{http://www.univie.ac.at/nonstatgab/toolbox.php}
    \item
	    Holighaus, D{\"o}rfler, Velasco and Grill, 2012: Based on the Velasco, Holighaus, Dörfler and Grill, 2011 - allowing perfect reconstruction. They propose sliCQT (slicing by using an overlapping window) to allow real-time computations. Code: \url{http://www.univie.ac.at/nonstatgab/toolbox.php}
    \item
	    Sch{\"o}rkhuber, Klapuri, Holighaus and Dörfler, 2014: Based on the Velasco, Holighaus, Dörfler and Grill, 2011 - allowing perfect reconstruction. They solve the problem with phases by means of a frequency mapping and they also propose a Variable-Q transform (that allows ie. ERBlets). Code: \url{http://www.cs.tut.fi/sgn/arg/CQT/}
    \item
	    Sch{\"o}rkhuber, Klapuri, Holighaus and D{\"o}rfler, 2014, a nonstationary Gabor transform, allows perfect reconstruction while the phases are still accurate. It might be interesting to implement this in Essentia.
\end{enumerate}

\todo[inline]{the original judith brown CQT is the origin of the research that led to the NSGT}

The Constant-Q transform (CQT) is time-frequency transform for musical signals, originally designed by \textcite{jbrown}, the relationship between the fundamental frequency and its harmonics on a logarithmic frequency scale more clearly than the linear frequency scale of the traditional discrete Fourier transform (DFT).

The original CQT had no inverse transform, but later works led to approximate inverses \cite{klapuricqt, fitzgeraldcqt}. 
, which has an important application in the perfectly-invertible CQT, or CQ-NSGT \cite{invertiblecqt}

The more general NSGT should be studied instead the CQT for the following reasons:
\begin{itemize}
	\item
		It solves the earlier CQT's \cite{jbrown, klapuricqt, fitzgeraldcqt} lack of stable inverse, which was a known weakness \cite{lackinverse}
	\item
		It can use other potentially interesting frequency scales besides the constant-Q logarithmic scale, such as the psychoacoustically-motivated mel, Bark, or ERB scales \todo{cite me}, or variable-Q scales \todo{cite gamma and other}
\end{itemize}

 The CQT has a high temporal resolution at high frequencies \cite{cqtransient} .

 A demonstration of the CQT is shown in figure \ref{fig:earlycqt}.

\begin{figure}[ht]
	\centering
	\subfloat[Linear frequency spectrum]{\includegraphics[height=4.75cm]{./images-tftheory/violindft.png}}
	\subfloat[Constant-Q transform]{\includegraphics[height=4.75cm]{./images-tftheory/violincqt.png}}
	\caption{Violin playing the diatonic scale, $G_{3} \text{(196Hz)} - G_{5} \text{(784Hz)}$}
	\label{fig:earlycqt}
\end{figure}

Additionally, the constant-Q transform \cite{jbrown, klapuricqt, invertiblecqt} even before its formulation as a specialized variant of the nonstationary Gabor transform \cite{balazs}, is an STFT applied with window of different sizes, which are of long duration at low frequencies to create a fine frequency resolution (and sacrificing time resolution as per the time-frequency uncertainty principle), and gradually decrease the windows in duration to improve the time resolution (and sacrifice frequency resolution). At the same time, consider that the iterative harmonic-percussive source separation algorithms in \cite{driedger, fitzgerald2} use two-pass spectral masking with two different configurations of spectrograms -- one with a large window size (4096 samples in \cite{driedger}, 16384 samples in \cite{fitzgerald2}) for representing the harmonic or pitched instruments sharply and estimating the harmonic mask, and one with a short window size (256 samples in \cite{driedger}, 1024 samples in \cite{fitzgerald2}) for representing percussion or transients more sharply and estimating the percussive mask.

The connection to the CQT, or NSGT, is that these contain within a single transform the high frequency resolution of a the large-size spectrogram in the low frequency regions, and the high time resolution of the small-size spectrogram in the high frequency regions. According to \textcite{musicsepgood}'s survey on music source separation, most spectral masking techniques try to exploit the 

\begin{figure}[ht]
	\centering
	\includegraphics[width=9cm]{./images-tftheory/tf_tradeoff_dorfler.png}
	\caption{Time-frequency tradeoff for a glockenspiel signal}
	\label{fig:dorflertradeoff}
\end{figure}

The time-frequency tradeoff is demonstrated on a musical glockenspiel signal in figure \ref{fig:dorflertradeoff}. Notice how the wide window spectrogram shows frequency components (horizontal lines) with a sharper definition than the blurry lines in the narrow window spectrogram, while the narrow window spectrogram shows temporal events (vertical lines) with a sharper definition than the wide window spectrogram.

\subsubsection{Nonstationary Gabor Transform (NSGT) and the sliCQ transform}
\label{sec:theorynsgt}

\todo[inline]{brief intro to frame theory and math stuff}

\todo[inline]{irregular time and frequency sampling, show grids, varying time-frequency resolution etc.}

\todo[inline]{arbitrary f scales and time scales}

\todo[inline]{slicq is the realtime variant}

\todo[inline]{whats the output, what does it mean, how does it relate to the FFT coefficients, time-frequency matrix}

\vfill
\clearpage

\subsection{Nonlinear frequency scales for music analysis}
\label{sec:freqscales}

\ichfeedback{better title? this is a bit long. ``Frequency scales that may be useful for music or psychoacoustic purposes'' seems like a mouthful - ``Frequency scales for music analysis``?}

\subsubsection{Scales based on Western pitch}

constant-q, log, western pitch scale, octave

variable-q - same with gamma offset

from \cite{variableq1, variableq2}, same as cq-log but with a gamma parameter

\subsubsection{Psychoacoustic scales}

mel, bark

\vfill
\clearpage

\subsection{Machine learning}
\label{sec:ml}

\ichfeedback{if i'm presenting a neural network, it's probably necessary to have this section?}

\subsubsection{Deep learning}
\label{sec:dl}

\vfill
\clearpage

\subsection{Music source separation}
\label{sec:musicsep}

\subsubsection{Task motivation and definition}

\todo[inline]{purposes and uses - why do we want to do this}

\subsubsection{Public datasets}

The most popular music stem dataset used by SISEC and SigSep is the MUSDB18 dataset (\cite{musdb18}), and more recently the HQ (high-quality) version (\cite{musdb18hq}). MUSDB18-HQ contains stereo wav files sampled at 44100 Hz representing stems (drum, vocal, bass, and other) from a collection of permissively licensed music, specifically intended for recording, mastering, mixing (and in this case, ``de-mixing'', or source separation) research. It combines earlier mixing/demixing datasets (\cite{otherdataset1, otherdataset2}).

The songs in the MUSDB18-HQ dataset have a fixed train, validation, and test split. Following the rules defined in the ISMIR 2021 Music Demixing Challenge,\footnote{\url{https://www.aicrowd.com/challenges/music-demixing-challenge-ismir-2021}} for a network to be considered trained only on MUSDB18-HQ, the predefined data splits must be used.

\subsubsection{Evaluation measures}

The SigSep\footnote{\url{https://sigsep.github.io/}} community, borrowing from the methodology of Signal Separation Evaluation Campaign (SISEC), uses the BSS (Blind Source Separation) Eval \cite{bss} objective measure for separation quality. There are 4 distinct metrics that comprise BSS:

\begin{itemize}
\item
	\textbf{ISR:} source Image to Spatial distortion Ratio
\item
	\textbf{SIR:} Signal to Interference Ratio
\item
	\textbf{SAR:} Signal to Artifacts Ratio
\item
	\textbf{SDR:} Signal to Distortion Ratio
\end{itemize}

Out of these 4 scores, SDR is the single global score which is commonly used to summarize the overall performance of a music demixing system (\cite{sdruseful}). The SDR as it was defined in the ISMIR 2021 Music Demixing Challenge (and used to rank the participants) can be computed from the following equation:

\[ \]

In the SigSep community and in the most recent SiSec evaluation (\cite{sisec2018}), the BSS evaluation measure used is BSS v4, a variant of BSS available in their Python libraries museval\footnote{\url{https://github.com/sigsep/sigsep-mus-eval}} and bsseval.\footnote{\url{https://github.com/sigsep/bsseval}} The differences between BSS as used in SiSec 2016 (\cite{sisec2016}) and BSS v4 are outlined in the bsseval project's GitHub README file:

\begin{quote}
	One particularity of BSSEval is to compute the metrics after optimally matching the estimates to the true sources through linear distortion filters. This allows the criteria to be robust to some linear mismatches... this matching is the reason for most of the computation cost of BSSEval...

	For this package, we enabled the option of having time invariant distortion filters, instead of necessarily taking them as varying over time as done in the previous versions of BSSEval. First, enabling this option significantly reduces the computational cost for evaluation because matching needs to be done only once for the whole signal. Second, it introduces much more dynamics in the evaluation, because time-varying matching filters turn out to over-estimate performance. Third, this makes matching more robust, because true sources are not silent throughout the whole recording, while they often were for short windows
\end{quote}

\subsubsection{Survey of computational approaches}

\ichfeedback{spectral masking, NMF, machine learning, deep learning - i can lean on the machine learning introduction section right before}

\todo[inline]{summary of approaches over the year e.g. nonnegative matrix factorization to machine learning to deep learning}

\subsubsection{Time-frequency masking and oracle estimators}

\ichfeedback{i think the idea of the oracle mask computed from ground truths is important enough to be in the section title}

\ichfeedback{it will come up later in the thesis when choosing hyperparameters for the sliCQ}

\subsubsection{Open-Unmix (UMX) and CrossNet-Open-Unmix (X-UMX)}

 \textcite{umx}'s deep learning model for music source separation is intended to be a near state-of-the-art, open implementation based on the open MUSDB18 and MUSDB18-HQ datasets and designed to foster source separation research \cite{musdb18, musdb18hq}. A deep neural network is used to estimate the magnitude spectrograms of the sources given a mixed song as an input. The sources are the same as the four stems per track in MUSDB18: drums, vocals, bass, other. Finally, the estimate is used to compute a soft mask.

 \subsubsection{Convolutional denoising autoencoders}

 \todo[inline]{plumbley's papers here and a bit of CNN theory}

\end{document}
