\documentclass[letter,12pt]{article}
\usepackage[left=2cm, right=2cm, top=2cm, bottom=2cm]{geometry}
%\setlength{\parindent}{0pt}
%\usepackage[shortlabels]{enumitem}
%\usepackage{graphicx}
%\usepackage{amsmath}
%\usepackage{amssymb}
%\usepackage{verbatim}
%\usepackage{listings}
%\usepackage{minted}
%\usepackage{subfig}
\usepackage{todonotes}
\usepackage{hyperref}
\hypersetup{
    colorlinks,
    citecolor=black,
    filecolor=black,
    linkcolor=black,
    urlcolor=black
}
\usepackage{setspace}
\renewcommand{\topfraction}{0.85}
\renewcommand{\textfraction}{0.1}
\renewcommand{\floatpagefraction}{0.75}
\usepackage[
    %backend=biber, 
    natbib=true,
    style=numeric,
    sorting=none,
]{biblatex}
\addbibresource{citations.bib}
\newcommand\ThesisTitle{Music Source Separation with the sliCQ Transform}


\newenvironment{tight_enumerate}{
\begin{enumerate}
  \setlength{\itemsep}{0pt}
  \setlength{\parskip}{0pt}
}{\end{enumerate}}

\title{\ThesisTitle}
\author{Sevag Hanssian, sevag.hanssian@mail.mcgill.ca}

\begin{document}

\maketitle

\section{Introduction / Motivation}

The short-time Fourier transform (STFT) is an important tool for the time-frequency analysis of acoustic signals. It is computed by applying the discrete Fourier transform on overlapping, fixed-size windows of the input signal, and it is commonly used to represent music in digital signal processing algorithms and machine learning models for music information retrieval (MIR). Despite its popularity, the STFT limited by a fixed and bounded time-frequency resolution determined by the window size. In the sliCQ transform \cite{invertiblecqt}, which is the realtime variant of the Nonstationary Gabor Transform (NSGT) \cite{balazs}, the Fourier transform is applied with varying window sizes to adapt the time-frequency resolution to the characteristics of the studied signal. A well-known use of the sliCQ transform is to implement the constant-Q transform for music \cite{jbrown, klapuricqt}.

In the task of music source separation \cite{musicsepgood}, a mixed song is decomposed into its constituent sources, which are typically either different instruments (drums, bass, vocals) or groups of sources with similar characteristics (percussive, harmonic). Machine learning models for music source separation with the STFT have achieved success in recent years \cite{sisec2018}, but choosing the appropriate time-frequency resolution for the STFT is an important step which can significantly affect results \cite{tftradeoff1, tftradeoff2}. In this thesis, the STFT is replaced with the sliCQ transform in Open-Unmix (umx), a contemporary neural network \cite{umx} for music source separation. The proposed model, \textit{umx-sliCQ}, was submitted to the ISMIR 2021 Music Demixing Challenge,\footnote{\url{https://www.aicrowd.com/challenges/music-demixing-challenge-ismir-2021}} where it successfully separated the hidden data set within the required time limit. The submitted model did not surpass STFT-based submissions, but demonstrates a viable and flexible starting point for future sliCQ-based models.

\section{Previous Work}

\citet{umx}'s Open-Unmix is an open-source, reference implementation of a deep neural network for music source separation, based on the STFT and intended to foster reproducible research. It was chosen as the initial model to adapt to the sliCQ transform. Other music source separation algorithms exist which use multiple STFTs with different window size \cite{fitzgerald1, driedger}, or which replace the STFT with the constant-Q transform \cite{fitzgerald2}. In both cases, the aim is to improve the results by manipulating time-frequency resolution. \citet{plumbley2} presented a multi-resolution neural network for singing voice separation in music. In their model, ``multi-resolution'' refers to time-frequency resolution, which is varied by applying different sizes of time and frequency filters to the STFT.

\section{Proposed Research / Methodology}

The thesis will be broken up into the following sections:
\begin{tight_enumerate}
	\item
		Present the motivation for why time-frequency analysis is useful for acoustic signals, how the STFT is typically used in music processing systems, and its limitations.
	\item
		Describe the constant-Q transform, NSGT, and sliCQ transforms, and how they use varying time-frequency resolution to better represent musical signals. Show why the NSGT and sliCQ are good implementation choices for the constant-Q transform.
	\item
		Describe the implementation of tensor operations in the sliCQ transform using PyTorch \cite{pytorch}, enabling its efficient and optimized computation in deep learning models.
	\item
		Describe the task of music source separation, as well as typical STFT-based approaches. Present Open-Unmix and describe its architecture, codebase, and baseline performance.
	\item
		Describe how Open-Unmix was adapted step by step to replace the STFT with the sliCQ transform, and show the training details and performance of the final trained model (as well as important intermediate steps such as hyperparameter tuning).
\end{tight_enumerate}

\section{Contributions / Summary}

In music source separation approaches that use the STFT, choosing the appropriate time-frequency resolution plays an important role. The sliCQ transform with varying time-frequency resolution is explored as a replacement for the STFT. First, the reference Python implementation for the sliCQ transform\footnote{\url{https://github.com/grrrr/nsgt}} was forked\footnote{\url{https://github.com/sevagh/nsgt}} to add support for fast and efficient GPU computation using PyTorch. This first contribution enables the sliCQ transform to be used by other researchers in different GPU-based machine learning models and problems going forward.

Next, the sliCQ transform was used in a deep neural network for music source separation, using the STFT-based Open-Unmix as a starting point. The proposed model, \textit{umx-sliCQ}, was submitted to the ISMIR 2021 Music Demixing Challenge. It passed the test suite, showing that the sliCQ transform can be used in an effective music source separation system. The final submitted model did not surpass the score of the original Open-Unmix model. However, the performance of \textit{umx-sliCQ} was shown to improve significantly after several rounds of hyperparameter tuning. Future work should build on this and explore more performance gains from further model architecture changes or hyperparameter tuning.

\vfill
\clearpage %force a page break

%\nocite{*}
\printbibheading[title={References},heading=bibnumbered]
\printbibliography[heading=none]

\vfill
\clearpage %force a page break

\end{document}
