\documentclass[usenames,dvipsnames]{beamer}
\usetheme{Boadilla}
\usepackage{hyperref}
\usepackage{graphicx}
\usepackage{multimedia}
\usepackage{fancyvrb}
\usepackage{soul}
\usepackage{multicol}
\usepackage{optparams}
\usepackage{adjustbox}
\usepackage{tikz}
\usetikzlibrary{shapes,positioning}
\newcommand{\foo}{\hspace{-2.3pt}$\bullet$ \hspace{5pt}}
\usepackage{subfig}
\usepackage[
    backend=biber,
    natbib=true,
    style=numeric,
    sorting=none,
    style=verbose-ibid,
    maxcitenames=1, %remove this outside of toy presentations
]{biblatex}
\addbibresource{citations.bib}
\usepackage{pgfpages}
\usepackage{xcolor}
\definecolor{ao(english)}{rgb}{0.0, 0.5, 0.0}
\definecolor{burgundy}{rgb}{0.5, 0.0, 0.13}
%\setbeameroption{show notes}
%\setbeameroption{show notes on second screen=right}
\setbeameroption{hide notes}

\title{Better music demixing with sliCQT}
\subtitle{Submission to Cadenza Challenge CAD1}
\author{Sevag Hanssian}
\date{December 08, 2023}
\setbeamertemplate{navigation symbols}{}

\AtEveryBibitem{%
  \clearfield{pages}%
  \clearfield{volume}%
  \clearfield{number}%
  \clearlist{journal}%
  \clearfield{booktitle}%
}

\renewbibmacro{in:}{}

\AtEveryCitekey{%
  \clearfield{pages}%
  \clearfield{volume}%
  \clearfield{number}%
  \clearfield{doi}%
  \clearfield{journal}%
  \clearlist{journal}%
  \clearfield{booktitle}%
  \clearfield{isbn}%
  \clearfield{title}%
  \clearfield{url}%
\ifentrytype{article}{
    \clearfield{journal}%
}{}
\ifentrytype{inproceedings}{
    \clearfield{booktitle}%
}{}
}

\begin{document}

\begin{frame}
\maketitle
\end{frame}

\begin{frame}
	\frametitle{Time-frequency tradeoffs}
	Median-filtering harmonic/percussive source separation (HPSS)\footfullcite{fitzgerald1, driedger}
	\begin{figure}[ht]
		\centering
		\vspace{-0.5em}
		\includegraphics[height=2.3cm]{./images/hpss.png}
		\includegraphics[height=2.3cm]{./images/hpss_harm.png}
		\includegraphics[height=2.3cm]{./images/hpss_perc.png}
		\vspace{-0.5em}
	\end{figure}
	\begin{enumerate}
        \item
                Short window (256) for percussion, long window (4096) for harmonic
	\item
		Short-time Fourier Transform (STFT) window size matters per-target\footfullcite{tftradeoff2} in VDBO problems
	\item
		In musical and auditory contexts, frequency resolution should increase from high to low frequencies (vice-versa for time resolution)\footfullcite{cqtransient, doerflerphd}
	\item
		CQT\footcite{jbrown} uses long windows in low frequencies and short windows in high frequencies for the 12-tone Western pitch scale
	\end{enumerate}

\end{frame}

\begin{frame}
	\frametitle{sliCQT vs. STFT}
	\begin{enumerate}
	\item
		Nonstationary Gabor Transform (NSGT)\footcite{balazs}, realtime sliCQ Transform\footcite{invertiblecqt, slicq, variableq1}
	\item
		STFT-like transforms with windows that vary with time
	\item
		CQT motivates the NSGT/sliCQ, but can use any monotonically increasing frequency scale (log/cq, mel, Bark, etc.)
	\item
		Outputs the familiar Fourier coefficients with perfect inverse
	\end{enumerate}
	\begin{figure}[ht]
		\centering
		\vspace{-0.5em}
		\includegraphics[height=3.5cm]{./images/slicq_spectral.png}
		\vspace{-0.5em}
		\caption{sliCQ: 262-bin Bark scale, 32.9--22050 Hz}
		\vspace{-0.5em}
	\end{figure}
\end{frame}

\begin{frame}
	\frametitle{xumx-sliCQ v1 @ MDX 2021}
        Use sliCQT (Bark scale, 262 bins, 32.9--22050 Hz) and convolutional autoencoder architecture\footfullcite{plumbley1} to achieve 3.6 dB signal-to-distortion ratio (SDR): \url{https://github.com/sevagh/xumx-sliCQ/tree/v1}
	\begin{figure}[ht]
		\centering
		\includegraphics[height=3cm]{./images/gspi_overlap_flatten.png}
		\includegraphics[height=3cm]{./images/gspi_overlap_proper.png}
		\includegraphics[height=2.5cm]{./images/xumx_slicq_pertarget.png}
		\vspace{-0.5em}
		\caption{xumx-sliCQ: block diagram}
		\vspace{-0.5em}
	\end{figure}
\end{frame}

\begin{frame}
	\frametitle{xumx-sliCQ v2 @ CAD1 2023}
        \begin{enumerate}
        \item
                Bark scale may have some benefits for human listeners
        \item
                Focused solely on VDBO demixing problem
        \item
                \textbf{4.4 dB SDR} up from 3.6, including better handling of overlap-add, mask sum loss, differentiable Wiener filtering, and complex MSE\footfullcite{dannasep}:  \url{https://github.com/sevagh/xumx-sliCQ/tree/v2}
        \end{enumerate}
        \vspace{-1em}
        \begin{figure}[ht]
		\centering
		\includegraphics[height=3cm]{./images/slicq_overlap_improved.png}
	\end{figure}
        \vspace{-2em}
                \small \begin{align}
                        & x_{\text{mix}} = x_{\text{v}} + x_{\text{d}} + x_{\text{b}} + x_{\text{o}}, |X|_{\text{mix}} = M_{\text{v}}|X|_{\text{mix}} + M_{\text{d}}|X|_{\text{mix}} + M_{\text{b}}|X|_{\text{mix}} + M_{\text{o}}|X|_{\text{mix}} \nonumber \\
                        & \rightarrow 1 = M_{\text{v}} + M_{\text{d}} + M_{\text{b}} + M_{\text{o}} \nonumber
                \end{align} \normalsize
        \begin{enumerate}
        \setcounter{enumi}{3}
        \item
                \vspace{-2em}
                HAAQI score: mean of 0.094 vs. 0.255 of Baseline 1 (demucs)
        \item
                BAQ score: mean of 41.84 vs. 41.40 of Baseline 1 (demucs)
        \end{enumerate}
\end{frame}

\begin{frame}
	\frametitle{New demixing-related project}
        Aim of these systems are to improve listening experience for those with different hearing.\footnote{http://cadenzachallenge.org/about} VDBO models are not very accessible (inscrutable Python errors, need >64GB RAM, etc.)
        \vspace{0.5em}

        \url{https://freemusicdemixer.com}\ \\
        \vspace{0.7em}
        Optimized C++ inference for UMX + Demucs, compiled to WebAssembly, running in the web, client-side on your browser, under 4 GB of memory
        \begin{figure}[ht]
		\centering
		\includegraphics[height=3.75cm]{./images/freemusicdemixer2.png}
		\includegraphics[height=3.75cm]{./images/freemusicdemixer.png}
	\end{figure}
\end{frame}

\end{document}
